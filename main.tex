% !TeX root = ./main.tex
\documentclass[a4paper,xelatex,ja=standard]{bxjsarticle}
\usepackage{xeCJK}
\usepackage{amsmath, amsthm, amssymb, amsfonts}
\usepackage{bm}
\usepackage{braket}
\usepackage{biblatex}
\usepackage{url}
\addbibresource{bibliography/biblio.bib}
\title{量子力学の観測問題と哲学的ゾンビ(執筆中)}
\author{quantumtimemeasure@gmail.com}
\begin{document}
    \maketitle
    \input{0000_Introduction.tex}
    \input{1000_Epiphenomenalism.tex}
    \input{1100_PhilosophicalZombie.tex}
    \input{1200_EpiphenomenalismFormulation.tex}
    \input{1220_o7v5cmfl3dg.tex}
    \input{1230_puekkajc85o.tex}
    \input{1240_fvpo2fauh28.tex}
    \input{1250_a962dhgetk8.tex}
    \input{2000_MacroClassicalState.tex}
    \input{2200_ClassicalWorld.tex}
    \input{2210_MaterialPoint.tex}
    \input{2220_Temperature.tex}
    \input{2230_SecondLawOfThermodynamics.tex}
    \input{2240_FluidDynamics.tex}
    \input{2250_EulerFluidDynamics.tex}
    \subsubsection{電流と電荷}
電流、電荷については、電流が流れる導体そのものは動いていない場合で、電流についてEuler表現の場合には、時刻によらない宇宙の分割$\{V_p\}$を用いて、
\begin{equation}
    \label{eq:electric_current_state_euler}
    \bigotimes_p \ket{\rho^e_p(t),\bm{J}_p(t)} \in \bigotimes_p \mathcal{H}_p^a
\end{equation}
とかけるとする。ボリューム間での電荷の流出入が発生する。ここで、$\rho^e_p(t)$はボリューム$V(t)_p$の電荷密度、$\bm{J}_p(t)$はボリューム$V(t)_p$の電流密度である。
当然ながら、
\begin{equation}
    \bigotimes_p \ket{\rho^e_p(t),\bm{J}_p(t)} = e^{-iHt}\bigotimes_p \ket{\rho_p^e(0),\bm{J}_p(0)}
\end{equation}
である。\par
新解釈では、$\mathcal{H}^i$との相互作用を無視すると、マクロ・古典的な物理量の固有状態は、時間が進んでも古典・マクロ的な物理量の固有状態であり続けるとしている(そのように$\mathcal{H}^t$と$\mathcal{H}^i$を定義する)。従って、時刻$0$に電荷密度、電流密度について重ね合わせの状態でない(かつ異なるボリューム間でエンタングルしていない)状態は、時刻$t \neq 0$においても重なり合わせのない状態であり続けると考える。そのダイナミクスは、古典的な電磁気学に一致する。\par
電流が流れる導体そのものが動いている場合には、導体そのものについては、Lagrange表現を用いることとすると、時間の推移とともに変化する宇宙の分割$\{V(t)_p\}$があって、ボリューム間での物質全体の流出入はないが、電荷の流出入はあり、
\begin{equation}
    \label{eq:electric_current_state_lagrange}
    \bigotimes_p \ket{\rho_p(t),P_p(t),\rho^e_p(t),\bm{J}_p(t)} \in \bigotimes_p \mathcal{H}_p^a
\end{equation}
とかけるとする。脳細胞について考察するには、この表現が最も適していると考えられるため、この表現を今後基本的に用いる。
    \input{3000_Superposition.tex}
    \subsection{離散的な重ね合わせの状態になった質点の運動}
時刻$0$において、位置 $n \in \{p\}$にのみ、質量$M$、速度$\bm{V}$で飛行している質点がある状況を考えよう。その状態は
\begin{equation}
    \label{initial_material_point_state}
    \ket{0}_1 \otimes \ket{0}_2 \otimes \ldots \otimes \ket{M,\bm{V}}_n \otimes \ket{0}_{n+1} \ldots \in \bigotimes_p \mathcal{H}_p^a
\end{equation}
とかける。ここで、$\ket{0}$は真空の状態であり、$\ket{M,\bm{V}}$は質量$M$、速度$\mathbf{V}$で特定される状態である。\par
質点以外に$\ket{\phi} \in \mathcal{H}^i$が存在し、全体では状態は
\begin{equation}
    \ket{0}_1 \otimes \ket{0}_2 \otimes \ldots \otimes \ket{M,\bm{V}}_n \otimes \ket{0}_{n+1} \ldots \otimes \ket{\phi} 
\end{equation}
とする。$\ket{\phi}$との相互作用により、時刻$dt$において、\eqref{initial_material_point_state}は、
\begin{equation}
\begin{aligned}
    \ket{0}_{\{V(dt)_p\},1} & \otimes \ket{0}_{\{V(dt)_p\},2} \otimes \ldots \\ 
    & \otimes (\alpha \ket{M,\bm{V}+d\bm{V}_1}_{\{V(dt)_p\},n} + \beta \ket{M,\bm{V}+d\bm{V}_2}_{\{V(dt)_p\},n}) \\
    & \qquad \otimes \ket{0}_{\{V(dt)_p\},n+1} \ldots 
\end{aligned}
\end{equation}
になるとする。ただし、Lagrange表現を用いており、時刻により異なる宇宙の分割$\{V(t)_p\}$を用いてヒルベルト空間を分割していることを明示するように記法を変更している。また、$\alpha,\beta$は$|\alpha|^2+|\beta|^2=1$を満たす複素数である。時刻$t>dt$では$\ket{\phi}$との相互作用はなくなると仮定すると、2つの状態 
\begin{equation}
\begin{aligned}
    & \ket{0}_{\{V(dt)_p\},1}  \otimes \ket{0}_{\{V(dt)_p\},2} \otimes \ldots \\ 
    & \qquad \otimes (\ket{M,\bm{V}+d\bm{V}_1}_{\{V(dt)_p\},n} \\
    & \qquad \qquad \otimes \ket{0}_{\{V(dt)_p\},n+1} \ldots ,
\end{aligned}
\end{equation}
\begin{equation}
\begin{aligned}
    &\ket{0}_{\{V(dt)_p\},1}  \otimes \ket{0}_{\{V(dt)_p\},2} \otimes \ldots \\ 
    & \qquad \otimes (\ket{M,\bm{V}+d\bm{V}_2}_{\{V(dt)_p\},n} \\
    & \qquad \qquad \otimes \ket{0}_{\{V(dt)_p\},n+1} \ldots 
\end{aligned}
\end{equation}
中の質点は等速運動をすると考えられ、新解釈では重ね合わせの状態になることはないので、任意の時刻$t>dt$における状態は、
\begin{equation}
\begin{aligned}
    &\alpha \ket{0}_{\{V_1(t)_p\},1}  \otimes \ket{0}_{\{V_1(t)_p\},2} \otimes \ldots \\ 
    & \qquad \otimes (\ket{M,\bm{V}+d\bm{V}_1}_{\{V_1(t)_p\},n} \\
    & \qquad \qquad \otimes \ket{0}_{\{V_1(t)_p\},n+1} \ldots \otimes \ket{\phi'} \\ 
    & + \beta \ket{0}_{\{V_2(t)_p\},1}  \otimes \ket{0}_{\{V_2(t)_p\},2} \otimes \ldots \\ 
    & \qquad \otimes (\ket{M,\bm{V}+d\bm{V}_2}_{\{V_2(t)_p\},n} \\
    & \qquad \qquad \otimes \ket{0}_{\{V_2(t)_p\},n+1} \ldots \otimes \ket{\phi''} 
\end{aligned}
\end{equation}
となる。このように、時刻により異なる宇宙の分割$\{V(t)_p\}$を用いるだけでなく、$\{V_1(t)_p\}$と$\{V_2(t)_p\}$という異なる宇宙のボリュームへの分割を用いて表現した状態の重ね合わせ状態を考える(フォーミュレーションを行う)ことが、新解釈の一つのポイントとである(無矛盾に数学として定義できるかどうかまでは私には考察できないが)。
    \input{3200_SPDiffrentVolumes.tex}
    \input{3300_SPMaterialPointCollision.tex}
    \input{3400_SPMaterialPointCopenhagen.tex}
    \section{量子状態の随伴現象説}
新解釈においては、状態に随伴する意識・クオリアは、当該状態のマクロ・古典的な自由度(物理変数)にのみによって定まると考える。その前提となるのは、当該空間の状態の空間$\mathcal{H}$は、
\begin{equation}
    \mathcal{H} = \mathcal{H}^a \otimes  \mathcal{H}^t \otimes  \mathcal{H}^i
\end{equation}
とテンソル積に分解できるという考えである。ここで、$\mathcal{H}^a$は、マクロ・古典的自由度で指定される状態のヒルベルト空間であり、$\mathcal{H}^t$は熱平衡に達していると考えられるミクロな自由度で指定されるヒルベルト空間であり、$\mathcal{H}^i$は残りの自由度で指定される状態のヒルベルト空間である。
例えば、一人の人を含む範囲の空間の状態を$\ket{A}$ととし、その人に意識・クオリア$|a \rfloor$が随伴しているとする。すなわち、$\ket{A} \mapsto |a \rfloor$とする。
    \printbibliography
\end{document}