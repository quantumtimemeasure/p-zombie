% !TeX root = ./main.tex
\documentclass[a4paper,xelatex,ja=standard]{bxjsarticle}
\usepackage{xeCJK}
\usepackage{amsmath, amsthm, amssymb, amsfonts}
\usepackage{bm}
\usepackage{braket}
\usepackage{biblatex}
\usepackage{url}
\addbibresource{bibliography/biblio.bib}
\title{量子力学の観測問題と哲学的ゾンビ(執筆中)}
\author{quantumtimemeasure@gmail.com}
\begin{document}
    \maketitle
    \section*{はじめに}
本稿では、コペンハーゲン解釈、多世界解釈に代わる新しい量子力学の解釈を示そうと思う。新解釈に名前はまだない。新解釈は、波束の収束・射影仮説が不要な点で、多世界解釈に近い解釈であるが、遠い銀河を含めて世界が分かれるとは考えないことから、いわゆる多世界解釈とは異なる。EverettのもとのRelative-State Formulationにより近いものと考えられる。\par
本稿の検討は、「哲学的ゾンビ」のタイトルが示すように、哲学・形而上学に属するものであり、自然科学に属するものではない。本稿は、実験結果に関する限り、コペンハーゲン解釈に基づく量子力学(場の理論を含む)が正しいことを前提としている。そのため、新解釈によって、実験結果について新しい予言ができるようになることはない。異なる考え方があり、そのいずれが正しいか実験で確認しようがないとき、いずれの考え方を選ぶか考察することは、科学ではなく、哲学と私は考えている。したがって、本稿の考察は哲学である。自然科学の理論としては、コペンハーゲン解釈と新解釈・本稿の間に違いはなく、同じものである。\par
ただし、マクロな巨視的な状態に対しても、重ね合わせの原理が成り立つと考える。重ね合わせの原理が成り立つミクロな状態と重ね合わせのないマクロな状態の境界は知られていないため、これはコペンハーゲン解釈と矛盾するものでない。\par
本稿の構成(予定)は以下のとおりである。まず、新解釈は、随伴現象説に基づくものであるため、第1章で、まずその説明から始める。次に、古典的な世界がどのようにして量子力学、ヒルベルト空間から生み出されるのかの仮説を第2章及び第3章で述べる。そして、第4章で、その古典的な世界と随伴現象説の関係を述べる。第5章で、存在とは何かについての、私見を科学的実在論を中心に述べる。また、本稿はギリシア時代の原子論と同様に科学的な根拠がほぼ存在しない考察であるが、それが科学的な根拠を持つものになる可能性について整理して述べる。最後に第6章で、新解釈が倫理に与える影響について考察する。
    \section{随伴現象説}
随伴現象説とは、
\begin{quotation}
心の哲学において、物質と意識の間の因果関係について述べた形而上学的な立場のひとつで、『意識やクオリアは物質の物理的状態に付随しているだけの現象にすぎず、物質にたいして何の因果的作用ももたらさない』というもの。\\物質と意識を別の存在であると捉える二元論の立場を取りつつ、意識の世界で起こる反応には、必ずそれに対応する物質的反応が存在するという考え方である。\cite{wikipedia}
\end{quotation}
クオリアとは、
\begin{quotation}
「感じ」のことである。「イチゴのあの赤い感じ」、「空のあの青々とした感じ」、「二日酔いで頭がズキズキ痛むあの感じ」、「面白い映画を見ている時のワクワクするあの感じ」といった、主観的に体験される様々な質のことである。\par
外部からの刺激(情報)を体の感覚器が捕え、それが神経細胞の活動電位として脳に伝達される。すると何らかの質感が経験される。例えば波長700ナノメートルの光(視覚刺激)を目を通じて脳が受け取ったとき、あなたは「赤さ」を感じる。このあなたが感じる「赤さ」がクオリアの一種である。\cite{wikipediaa}
\end{quotation}
本稿においては、「実験装置の針が1を指していた」、「実験装置の針が2を指していた」という「感じ」もクオリアの一種であると考える。
    \subsection{哲学的ゾンビに対する本稿の立場}
本稿では、随伴現象説のなかでも、哲学的ゾンビは存在し得ないという立場をとる。すなわち、意識・クオリアが随伴する物質の状態には必ず意識・クオリアが随伴しており、同じ物質の状態にもかかわらず意識・クオリアが随伴したり随伴しなかったりはしないという立場である。\par
哲学的ゾンビとは、
\begin{quotation}
「物理的化学的電気的反応としては、普通の人間と全く同じであるが、意識(クオリア)を全く持っていない人間」\cite{wikipediab}
\end{quotation}
であり、意識・クオリアが随伴する物理状態には必ず意識・クオリアが随伴するのであれば、哲学的ゾンビは存在し得ない。物理的化学的電気的反応として普通の人間と全く同じであれば、意識・クオリアが随伴しているからである。
    \subsection{随伴現象説のフォーミュレーション}
\subsubsection{基本}
前述のように、物理的な状態は、意識・クオリアが随伴する状態と、随伴しない状態がある。しかも、意識・クオリアが随伴するかどうかは、物理的な状態によって定まっている。したがって、意識・クオリアが随伴する状態すべての集合 $\mathfrak{C}$ を定義することができる。すなわち、$|A\rangle \in \mathfrak{C}$であれば、状態$|A\rangle$は意識・クオリアを随伴している状態であり、$|B\rangle \notin \mathfrak{C}$であれば、状態$|B\rangle$は意識・クオリアを随伴していない状態である(前述したように、新解釈では、マクロな状態においても重ね合わせの状態が成り立つと考えるため、本稿では脳のようなマクロな状態もケットとして表現する)。\par
その一部に意識・クオリアが随伴する状態を含んでいれば、意識・クオリアがある状態と考えられるため(必ずしもそう考えるのが必須ではないが、本稿では、脳に意識・クオリアがあれば人間には意識・クオリアがあるという立場をとるため)、$|A\rangle = |C\rangle \otimes |d\rangle$とテンソル積に状態が分解でき、$|C\rangle\in\mathfrak{C}$であれば、$|A\rangle\in\mathfrak{C}$と本稿ではする。\par
$|C\rangle\in\mathfrak{C}$であれば、$|A\rangle\in\mathfrak{C}$とする背景には、脳が冗長性、可塑性を有していることがある。すなわち、テンソル積への分解の意味で、最小の意識・クオリアがある状態を定義することが難しい。例えば、$|A\rangle = |c\rangle \otimes |d\rangle \otimes |e\rangle$とテンソル積に状態が分解でき、$|c\rangle \otimes |d\rangle$も$|c\rangle \otimes |e\rangle$も意識・クオリアを随伴する状態であるが、$|c\rangle$のみでは意識・クオリアを随伴しない場合、意識・クオリを随伴する最小の状態を定義することが難しい。
    \subsubsection{意識・クオリアにかかる相互作用表示}
記載を簡便にするために、意識・クオリアにかかる相互作用表示を導入する。意識・クオリアは、脳細胞の物理的化学的電気的反応であり、静的なものではない。人間の脳がどのように意識・クオリアをエンコードしているかは解明されていないが、脳が外界の情報をエンコードするためにシンプルなものとして、Rate coding、Temporal coding、Population codingがあることが知られている\cite{wikipediac}。これらはすべて、ダイナミックなものである。しがって同一のクオリアが継続している間の状態であっても、シュレディンガー表示のもとでは、時刻により状態が異なってしまう。そのため、同一の意識・クオリアが継続している間同じ状態で表すことができるよう相互作用表示を用いることとする(正確には、そもそも、ダイナミックな変化に意識・クオリアは随伴しているのであり、特定の時刻の状態に意識・クオリアが随伴するわけではない)。\par
すなわち、ハミルトニアン$H$を、意識・クオリアの随伴にとって内在的な$H_c$とその他の部分$H_o$に下記のように分解し、
\begin{equation}
  H=H_c + H_o
\end{equation}
シュレディンガー表示の状態$|S(t)\rangle$と
\begin{equation}
  |I(t)\rangle = e^{iH_ct}|S(t)\rangle
\end{equation}
の関係にある相互作用表示の$|I(t)\rangle$を本稿では基本的に用いる。すなわち、
\begin{equation}
  |I(t)\rangle = e^{-iH_ot}|S(0)\rangle = e^{-iH_ot}|I(0)\rangle
\end{equation}
と、状態は$H_o$によってのみ時間発展する。
この表示であれば、時刻によらず同一の意識・クオリアが継続している間は、同じ状態で表すことができる。\par
若干単純化して(たぶん脳科学の観点からは誤って)、視覚の場合について表現するならば、$H_c$は脳の神経細胞間の相互作用のハミルトニアン、$H_o$は視覚細胞と脳の神経細胞間の相互作用と考えればわかりやすいだろう。眼球に入る光子が変化し視覚細胞から脳の神経細胞への入力が変われば、異なるクオリアが生じる。\par
上記の記載を脳科学の観点から誤っているとしたのは、視覚細胞からの入力の多くは、意識されることなく脳の神経細胞で処理され、意識がある状態で処理されるのは一部のみであることが知られているからである。したがって、$H_c$と$H_o$の間の区分は、視覚細胞と脳の神経細胞のハミルトニアンの違いではなく、脳の神経細胞内の違いであると考えるのが脳科学の観点からは正しいと考えられる。
    \subsubsection{意識・クオリアの相互作用表示の具体的イメージ例}
意識・クオリアにかかる相互作用表示をイメージすることは難しいため、ここでは、その例を具体的に示してみることにしたい。意識・クオリアがどのように物理的な状態に随伴しているかは、脳科学で現在のところ十分に解明されていないため、ここでの記載は、あくまで可能性の一つである。\par
例えば、\cite{Murray_2016}では、意識・クオリアの実体が、脳細胞の発火率の主成分であることが示唆されている。これは、神経細胞$i$の平均をゼロに調整した発火率を$r_i$とすると、ある$w_k^i$があって、意識・クオリアは$w_k^ir_i$に随伴しているということと理解できるだろう。ただし、ここで、$k\in\{1,2,\ldots,K\}$であり、同じ添字$i$については和を取るものとする。\par
すなわち、主成分の値の集合$P=\mathbb{R}^K$の分割$C_j,\,j\in\{0,1,2,\ldots,N\},\, P=\bigcup_j C_j,\, C_j \cap C_{j'}=\emptyset \, (j \neq j')$があって、$w_k^ir_i \in C_j$において、$j=0$の場合は意識・クオリアは随伴しておらず、$j \neq 0$の場合は、意識・クオリア$j$が随伴しているということである。ただしここでは、人間は有限の$N$種類の意識・クオリアを持ち得ると仮定した。\par
相互作用表示を用いたい趣旨は、各$C_j$の代表点$c_j \in C_j$をとって、物理的な状態を$|c_j,other \ physical \ quantities \rangle$のように書きたいということである。$other \ physical \  quantities$には、$w_k^ir_i - c_{j,k}$の値が含まれることになる。ここで、$c_{j,k}$は、$c_j \in \mathbb{R}^K$の第$k$成分の値である。\par
意識・クオリアの随伴にとって内在的な$H_c$では、随伴している意識・クオリアが変わることはないため、シュレディンガー表示においても
\begin{equation}
  e^{-iH_ct}|c_j,others \rangle = |c_j,others' \rangle 
\end{equation}
である。時間発展$e^{-iH_ct}$によって、意識・クオリアが異なる状態にはならないからである(それが意識・クオリアにとって内在的の定義であり、上式は定義により成り立つと考えるべきものである。意識・クオリアがワーキングメモリに基づくものだとすると、時間的に数秒しか継続しないため、時間の経過とともに $|c_0,others' \rangle \notin \mathfrak{C}$の状態になる必要があり、その変化は$e^{-iH_ot}$によりもたらされると考えることにする)。\par
しかし、各神経細胞$i$の位置$p$の電位の組$\{E_{i,p}\}$によって随伴する意識・クオリアは決まっていると考え、その状態を$|\{ E_{i,p}\},others \rangle$と記載することにするなら、シュレディンガー表示では同じ意識・クオリアが継続している間も、時間により状態は異なる電位の組$\{E'_{i,p}\}$によって表される$|\{ E'_{i,p}\},others' \rangle$になってしまう。一方、相互作用表示では、同じ意識・クオリアが継続している間は、電位の組$\{E_{i,p}\}$については同じ状態$|\{ E_{i,p}\},others' \rangle$と記載することができる。これが、相互作用表示を用いる理由である。\par
発火率の主成分によって随伴する意識・クオリアが定まるか否かは、現時点では仮説の一つの段階であり、必ずしも正しいとは限らない。しかし、各神経細胞$i$の位置$p$の電位の組$\{E_{i,p}\}$が存在することは、ほぼ科学的に確立しており、その値(通常的に物理量として観念されている値) を用いても同じ状態として記述できることに相互作用表示のメリットがある。
    \subsubsection{相互作用表示における不連続性}
相互作用表示においては、時間発展に伴い、意識・クオリアに関わる状態を指定する物理量、例えば$\{E_{i,p}\}$は、非連続に変化することに留意が必要である。これは、物理的な状態が非連続に変化するということではない。$others$を含めた状態$|\{ E_{i,p}\},others \rangle$は連続的に変化する。しかし、相互作用表示は、各$C_j$の代表点$c_j \in C_j$をとって、物理的な状態を$|c_j,others \rangle$のように書きたいという要請を別の方法で実現したものであるため、$others$を無視して$\{E_{i,p}\}$のみを見た場合には、不連続に変化するのである(この不連続性は量子論の不連続性・量子性とは全く関係のないものである)。\par
これは、実数$x \in \mathbb{R}$を整数$n \in \mathbb{Z}$とそこからの差分$x-n = y \in [0,1)$の組$(n,y)$として表現した場合、$x$の連続的な変化に対して、$n$が不連続に(実数の意味で不連続に)変化するようなものである。$n$が$\{E_{i,p}\}$に、$y$が$others$に相当する。$x = (n,y)$に非連続性がないように、状態$|\{ E_{i,p}\},others \rangle$に非連続性はない。

    \subsubsection{意識・クオリアのノーテーション}
本章の最後に、意識・クオリアのノーテションを導入しておこう。まず、意識・クオリアを$|\,\rfloor$の記号で表すことにする。そして、状態$|A\rangle$に意識・クオリアaが随伴していることを、$\mapsto$の記号を用いて、$|A\rangle \mapsto |a \rfloor$と表す。観測問題の焦点は、$|A\rangle \mapsto |a \rfloor$かつ$|B\rangle \mapsto |b \rfloor$のときに、$\alpha|A\rangle + \beta|B\rangle$にどのような意識・クオリアが随伴するのか、ということであるが、マクロ・古典的な状態の特性を示していないこの段階ではそれについて示すことは難しいので、後章に回すことにする。

    \section{マクロ・古典的な状態}
本性では、マクロ・古典的な状態の定義を行う。また、本章では特に記載しない場合、シュレディンガー表示を用いる。
\subsection{マクロに静止している場合の分割}
宇宙を分子の大きさより十分に大きく、しかし、ほぼ熱平衡に達していると考えられる程度には十分に小さいボリュームの集まり$\{V_p\}$に分割し、宇宙の状態のヒルベルト空間$\mathcal{H}$を$V_p$の状態のヒルベルト空間$\mathcal{H}_p$ のテンソル積に分割する(分子の大きさよりも十分に大きいとする必要はない可能性も高いが(いや、むしろ分子の大きさよりも小さくても以下の記載が成り立つとする必要性が高いが)、私も分子の粒子描像に毒されているので、ここではいったん分子の大きさよりも大きいとする)。すなわち、
\begin{equation}
    \mathcal{H} = \bigotimes_p \mathcal{H}_p
\end{equation}
とする(正確には、$\bigotimes_p \mathcal{H}_p$の元のほとんどは$\mathcal{H}$には含まれないため、分割とは異なるが、簡単のために分割の表現を用いる。フォック空間も、すべてが物理的にありえる状態ではなく、対象又は反対象な元のみが物理的な状態であるので、それと同様に$\bigotimes_p \mathcal{H}_p$の元のうち一部のみが物理的な状態に対応する)。\par
次に、各$\mathcal{H}_p$をマクロ・古典的とミクロのテンソル積へ分割することを考える。すなわち、マクロ・古典的変数で指定される状態のヒルベルト空間$\mathcal{H}_p^a$と残りの自由度のうち熱平衡に達していると考えられる自由度で指定されるヒルベルト空間$\mathcal{H}_p^t$と残りの自由度で指定される状態のヒルベルト空間$\mathcal{H}_p^i$のテンソル積に分割する($\mathcal{H}_p^t$の意味が不明と思われるが、説明が長くなるため、後で説明する)。
\begin{equation}
    \mathcal{H}_p = \mathcal{H}_p^a \otimes \mathcal{H}_p^t \otimes \mathcal{H}_p^i
\end{equation}
となる。ここで、
\begin{equation}
    \mathcal{H}^i = \bigotimes_p \mathcal{H}_p^i,
\end{equation}
\begin{equation}
    \mathcal{H}^t = \bigotimes_p \mathcal{H}_p^t
\end{equation}
を導入すると、
\begin{equation}
    \mathcal{H} = \bigotimes_p \mathcal{H}_p^a \otimes  \mathcal{H}^t \otimes  \mathcal{H}^i
\end{equation}
となる。\par
ハミルトニアン$H$は、以下のように分割されると考える。
\begin{equation}
H=\sum_pH_p^a + H^t + H^i + \sum_p H_p^i + \sum_{p \neq q} H^{int}_{p,q}
\end{equation}
それぞれの意味は、記号から想像されるとおりであり、一部説明すると、$\sum_p H_p^i$は熱平衡にないミクロな状態とマクロ・古典的な状態の相互作用である。例えば、空気中を飛行する電子と電子の測定器との相互作用がこれに該当する。空気中を飛行する電子は、マクロな状態である空気と熱平衡の状態にはないと考えられる。$\sum_{p \neq q}H^{int}_{p,q}$は異なる位置のマクロ・古典的な自由度(物理変数)間の相互作用であり、古典物理学における通常の相互作用である(ただし、主たる内容は異なる位置への流入、流出であり、いわゆる相互作用ではないことに注意が必要である)。そのため、本来は$\sum_{p \neq q}H^{a \, int}_{p,q}$と記載すべきと考えられるが、$a$の記載を省略したものである。本稿では、相対論的量子力学を考えているので、近接相互作用しかなく、異なるボリューム間の相互作用は、表面での相互作用に限られるとするが、非相対論的な量子力学を考えることもできる。\par
$\sum_{p \neq q} H_{p,q}^{t\,int}$の項が明示的に無いのは、熱平衡にある状態の相互作用は、ランダムさがあり、それを$H^t$で考慮するならば、ボリューム間の相互作用は、マクロ・古典的な相互作用に集約できると考えられるためである。マクロ・古典的な自由度には、温度も含まれ(エネルギーを通して間接的に含まれると考えることもできる)、異なる温度のボリュームの接触に基づく温度の均一化は、$\sum_{p \neq q} H^{int}_{p,q}$の方に含まれる。そのため、$\sum_{p \neq q} H_{p,q}^{t\,int}$の項目は不要としてよいと見込まれる(後述する)。
    \subsection{古典世界}
前節のフォーミュレーションにおいては、いわゆる古典極限($\hbar$を$0$とした極限)は、以下の2式となる。
\begin{equation}
    \mathcal{H} = \bigotimes_p \mathcal{H}_p^a,
\end{equation}
\begin{equation}
H=\sum_pH_p^a+ \sum_{p \neq q} H^{int}_{p,q}.
\end{equation}
新解釈では、これを古典世界と呼ぶことにする。極限をとったわけではなく、$\bigotimes_p \mathcal{H}_p^a \otimes  \mathcal{H}_p^t \otimes  \mathcal{H}_p^i$から$\bigotimes_p \mathcal{H}_p^a$だけを抜き出したものだからである(ちなみに、$\mathcal{H}_p^t \otimes  \mathcal{H}_p^i$の影響が小さいから無視したと考えれば、極限をとったことになる。新解釈のポイントは、影響が小さいから無視できるのでなく、それらの影響が大きくても、古典世界を分割して取り出すことができると考えるところにある)。これを「宇宙から古典世界を抜き出した」と表現することにしよう。古典世界のポイントは、上記のハミルトニアン$H$による時間発展のもとでは、マクロ・古典的な自由度(物理変数)の固有状態は、当該物理変数の固有状態であり続け、重ね合わせの状態にならないということである(これは近似的にそうなだけである可能性がある。また奇異に聞こえるかもしれないが、量子力学が不要であれば、物理変数の値は運動方程式により決定論的に(ユニークに)決まるという極めて常識的なことを別の表現で述べているにすぎない)。古典世界の状態は重ね合わせの状態にならないというわけではなく、重ね合わせの状態でなければ、時間が進んでも重ね合わせの状態ではないということに留意する必要がある。
    \subsubsection{古典極限における質点運動}
最もシンプルなケースとして、位置 $n \in \{p\}$にのみ、質量$M$、速度$\mathbf{V}$、逆温度$\beta$で飛行している質点がある状況を考えよう。その状態は
\begin{equation}
    |0\rangle_1 \otimes |0\rangle_2 \otimes \ldots \otimes |M,\mathbf{V},\beta \rangle_n \otimes |0\rangle_{n+1} \ldots \in \bigotimes_p \mathcal{H}_p^a
\end{equation}
とかける。ここで、$|0\rangle$は真空の状態であり、$|M,\mathbf{V},\beta \rangle$は質量$M$、速度$\mathbf{V}$、逆温度$\beta$で特定される状態である。\par
上記を時刻$0$における状態とし、その時点では位置$n$は原点にあったとする。時刻$t$においては、位置$n$が$\mathbf{V}t$にある宇宙の分割$\{V(t)_p\}$を用いてヒルベルト空間を分割して、
\begin{equation}
    \mathcal{H} = \bigotimes_p \mathcal{H}^a(t)_p
\end{equation}
とすると、時刻$t$の状態は同じく、
\begin{equation}
    |0\rangle_{(t)1} \otimes |0\rangle_{(t)2} \otimes \ldots \otimes |M,\mathbf{V},\beta \rangle_{(t)n} \otimes |0\rangle_{(t)n+1} \ldots \in \bigotimes_p \mathcal{H}^a(t)_p
\end{equation}
とかける。そして、それはハミルトニアン$H$による時間発展であるから、
\begin{equation}
\begin{aligned}
    e^{-iHt}|0\rangle_1 \otimes |0\rangle_2 \otimes \ldots \otimes |M,\mathbf{V},\beta \rangle_n \otimes |0\rangle_{n+1} \ldots \\
    =  |0\rangle_{(t)1} \otimes |0\rangle_{(t)2} \otimes \ldots \otimes |M,\mathbf{V},\beta \rangle_{(t)n} \otimes |0\rangle_{(t)n+1} \ldots
\end{aligned}
\end{equation}
である。\par
時刻により異なる宇宙の分割$\{V(t)_p\}$を用いず、同じ分割$\{V_p\}$を用いるならば、
\begin{equation}
\begin{aligned}
    e^{-iHt}|0\rangle_1 \otimes |0\rangle_2 \otimes \ldots \otimes |M,\mathbf{V},\beta \rangle_{n(0)} \otimes |0\rangle_{n(0)+1} \ldots \\
    =  |0\rangle_{1} \otimes |0\rangle_{2} \otimes \ldots \otimes |M,\mathbf{V},\beta \rangle_{n(t)} \otimes |0\rangle_{n(t)+1} \ldots
\end{aligned}
\end{equation}
となる。ただしここで、$V_{n(0)}$は原点を含むボリュームであり、$V_{n(t)}$は、$\mathbf{V}t$の位置を含むボリュームである。\par
質量$M$、速度$\mathbf{V}$、逆温度$\beta$の質点と質量$M'$、速度$\mathbf{V}'$、逆温度$\beta$の質点が衝突し、それぞれの質点の速度が速度$\mathbf{V}''$及び$\mathbf{V}'''$になったとすると、
\begin{equation}
\begin{aligned}
    e^{-iHt}|0\rangle_1 \otimes \ldots \otimes |M,\mathbf{V},\beta \rangle_{n(0)} \otimes \ldots \otimes |M',\mathbf{V}',\beta \rangle_{m(0)} \otimes \ldots \\
    =  |0\rangle_{1} \otimes \ldots \otimes |M,\mathbf{V}'',\beta \rangle_{n(t)} \otimes \ldots
    \otimes |M',\mathbf{V}''',\beta \rangle_{m(t)} \otimes \ldots 
\end{aligned}
\end{equation}
となる。ただし、時刻$t$において、それぞれの質点は、$V_{n(t)}$及び$V_{m(t)}$に存在するとする。\par
このように、複数の質点が存在するとしても$H=\sum_pH_p^a+ \sum_{p \neq q} H^{int}_{p,q}$による時間発展により、マクロ・古典的物理変数の固有状態は、その固有状態であり続ける。これは、質点の数が増えても、質点同士が接合して一体として運動するようになっても同じである。
    \subsubsection{温度について}
新解釈では、マクロ・古典的物理変数の固有状態は、ハミルトニアンから、熱平衡にないミクロな状態とマクロ・古典的な状態の相互作用$\sum_pH_p^i$を除いたものによる時間発展では、マクロ・古典的物理変数の固有状態であり続けると考える(これと、随伴現象説の組み合わせが、コペンハーゲン解釈の射影仮説に代わるものである)。マクロ・古典的物理変数には、温度も含まれる。従って、温度についても重ね合わせ状態にはならない。その説明のため、ここでは、宇宙の分割のうち、時刻$t=0$において接触した${V_1,V_2}$のみを考慮することにしよう。 したがって、
\begin{equation}
    \mathcal{H} = \mathcal{H}_1^a \otimes \mathcal{H}_2^a \otimes \mathcal{H}_1^t \otimes \mathcal{H}_2^t 
\end{equation}
となる。また、逆温度以外のマクロ・古典変数は記載を省略することにする。時刻$0$おいて、$V_1$の逆温度は$\beta_1$、$V_2$の逆温度は$\beta_2$であったとする。時間の進展とともに、両ボリュームの温度は同じ値$\beta$になる。すなわち、
\begin{equation}
    \beta_1(0) = \beta_1,
\end{equation}
\begin{equation}
    \lim_{t \to \infty} \beta_1(t) = \beta,
\end{equation}
\begin{equation}
    \beta_2(0) = \beta_2,
\end{equation}
\begin{equation}
    \lim_{t \to \infty} \beta_2(t) = \beta,
\end{equation}
を満たす関数$\beta_1(t)$、$\beta_2(t)$が存在し、
\begin{equation}
\label{eq:thermal_time_evolution}
\begin{aligned}
    e^{-iHt}|\beta_1 \rangle_1 \otimes |\beta_2 \rangle_2 \otimes \sum_{v \in \{v_i\}} \sigma_1(v) |v\rangle_1 \otimes \sum_{v \in \{v_i\}}  \sigma_2(v) |v\rangle_2\\
    =  |\beta_1(t) \rangle_1 \otimes |\beta_2(t)\rangle_2 \otimes   \sum_{v \in \{v_i\} ,\, w \in \{v_i\}} \sigma(t,v,w)|v\rangle_1 \otimes |w\rangle_2
\end{aligned}
\end{equation}
となる。ただし、簡単のために$\mathcal{H}_1^t = \mathcal{H}_2^t$とし、その正規直行基底を離散的な$\{|v_i\rangle\}$とした。離散的とはかぎらないため、その際は積分への読み替えが必要である。また、$\beta_1(t),\,\beta_2(t),\,\sigma(t,v,w)$は、正確には$\sigma_1,\sigma_2$の汎関数(離散的であるが汎関数と呼ぶなら)であるから、それぞれ$\beta_1[\sigma_1,\sigma_2](t),\,\beta_2[\sigma_1,\sigma_2](t),\,\sigma[\sigma_1,\sigma_2](t,v,w)$と書くのが本来である。\par
  しかし、実際の実験では、$\beta_1(t),\,\beta_2(t)$は実験により異なることはなく、同じ時間推移となる。$\beta_1[\sigma_1,\sigma_2](t),\,\beta_2[\sigma_1,\sigma_2](t)$と$\sigma_1,\sigma_2$の汎関数となるなら、未知の物理量に依存するため、実験のたびに$\beta_1(t),\,\beta_2(t)$の時間推移は異なるはずである。したがって、実験結果からは、$\sigma_1,\sigma_2$の汎関数ではないと考えられる。マクロ・古典的な物理量の推移は、熱平衡にある状態が具体的にどの状態でも、その違いに依存しないのである。新解釈では、つねにこれが成り立つと仮定する(より正確には、これが熱平衡にあることの定義であり、マクロ・古典的自由度を除いた残りの自由度を$\mathcal{H}_p^t \otimes \mathcal{H}_p^i$と熱平衡ヒルベルト空間と非熱平衡ヒルベルト空間のテンソル積に分解する際の基準である)。
    \subsubsection{熱力学の第二法則が成り立つ理由}
観測問題という本稿のテーマとは少し離れるが、ここで、熱力学の第二法則、すなわち熱が温度の高い方から低い方に流れる理由について、仮説を述べておこう。熱力学の第二法則は、一般的に時間反転対称性を破っていると考えられているが、新解釈では、量子力学が正しい(量子電磁力学QEDは時間反転対称性を持つ)としているため、時間反転対称性は成り立っていると考える。時間反転演算子を$T$として、時間反転対称性は、
\begin{equation}
\label{eq:temperature_synmetry}
    \beta_p(t) = \beta_p(-t), \; p \in \{1,2\},
\end{equation}
\begin{align}
\label{eq:time_reversal}
 T|\beta_1(t) \rangle_1 \otimes |\beta_2(t)\rangle_2 \otimes   \sum_{v \in \{v_i\} ,\, w \in \{v_i\}} \sigma(t,v,w)|v\rangle_1 \otimes |w\rangle_2 \\
 =|\beta_1(t) \rangle_1 \otimes |\beta_2(t)\rangle_2 \otimes   \sum_{v \in \{v_i\} ,\, w \in \{v_i\}} \sigma(-t,v,w)|v\rangle_1 \otimes |w\rangle_2
\end{align}
であれば、成り立つと考えられる。温度は時間反転により変わらないと考えられるため(分子の速度を全て反転しても温度は変わらないと考えられているため)、全ての$t$について、$T|\beta_p(t) \rangle_p = |\beta_p(t) \rangle_p$としている。ここで、\eqref{eq:temperature_synmetry}のように$\beta_p(t) = \beta_p(-t)$であれば、時刻$t$の状態を時間反転した状態、すなわち\eqref{eq:time_reversal}は、$|\beta_1(-t) \rangle_1 \otimes |\beta_2(-t)\rangle_2 \otimes \sum_{v \in \{v_i\} ,\, w \in \{v_i\}} \sigma(-t,v,w)|v\rangle_1 \otimes |w\rangle_2$となる。これは、時刻$-t$での状態である。従って、当該時刻から時間が$t$進むと、時刻$0$での状態となる。従って、時間反転した状態であることが確かめられる。より一般的には、時刻$t$の状態を$|t\rangle$、それを\eqref{eq:time_reversal}により時間反転した状態を$|Tt\rangle$と書けば、
\begin{equation}
e^{-iHt'}|Tt\rangle = |t-t'\rangle
\end{equation}
となっていることが確かめられる。これは、正しく時間反転状態となっているということである。\par
逆は必ずしも真ではないので、時間反転状態が\eqref{eq:time_reversal}であるとは限らないが、時間反転についての考察は本稿の主題ではないので、本稿では\eqref{eq:time_reversal}が時間反転状態であると考えることにする。\par
以上の考察から、時刻$0$において接触した状態に、時刻$t<0$から接触したままの時間発展でなったとすると、それまでの間、すなわち$t<0$において、接触した領域間の温度差は広がっていくことがわかる。$\beta_p(t) = \beta_p(-t)$だからである。これが理論的な考察結果であるが、残念ながら、それを実験で確かめることはできない。時刻$0$に接触させる以外に、時刻$0$と同じ状態を準備する方法を我々は知らないからである。ハミルトニアンによる時間発展においては、時間反転の対称性は保たれているが、それを確かめる実験を行うことはできないというのが、熱力学の第二法則が時間反転対称性を破っているようにみえる理由である。\par
時刻$0$の特殊性は、$\mathcal{H}_1^a \otimes \mathcal{H}_2^a \otimes \mathcal{H}_1^t \otimes \mathcal{H}_2^t$というテンソル積で表されるヒルベルト空間において、エンタングルのまったくない状態ということである。この時に、温度差は最大になる。一方、$0$以外の時刻では、$\sum_{v \in \{v_i\} ,\, w \in \{v_i\}} \sigma(t,v,w)|v\rangle_1 \otimes |w\rangle_2 $の式が示すようにエンタングル状態にある。このことから、エンタングルのない状態から時間発展又は時間後進すると温度差が小さくなると想像される。\par
別の観点から熱力学の第二法則が成り立つ理由を理解しようとするならば、ヒルベルト空間$\mathcal{H}$には、時間発展に伴い温度差が縮小する部分空間$\mathcal{H}^+$、時間発展に伴い温度差が拡大する部分空間$\mathcal{H}^-$、もともと温度差がなく時間発展で温度差が生じない$\mathcal{H}^\infty$があり、熱力学の第二法則は、「人間が始状態として用意できるのは部分空間間$\mathcal{H}^+$の要素の状態だけである。」と表すことができるだろう。ただし、ここで完全に同じ温度の状態(誤差のない状態)を用意することはできないので$\mathcal{H}^\infty$を用意することはできないとした。正確には、時間発展により$\mathcal{H}^\infty$の要素に状態がなることもないと考えられる。\par
そうすると、次の疑問は、なぜ人間には$\mathcal{H}^+$の状態しか用意できないのかということになる。これは、観測した状態しか操作できないためと思われる(より正確には、観測した物と相互作用した物しか操作できないためと思われる)。観測していない物体を人は動かすことができないことは、ある意味当然であろう(哲学としては、その理由が不思議で検討する必要があるということになり、哲学である本稿はそれを検討する必要があるという立場であるが、ここではそれについては不問にしておきたい)。どこにあるかも把握していない(=観測していない)物体を動かすことはできない。したがって、「なぜ人間には$\mathcal{H}^+$の状態しか用意できないのか?」という疑問は、「量子力学における観測とは何か?」「観測によって波束が収束するのはなぜか?」という量子力学の観測問題と極めて強く関わってくる。したがって、本稿の主題である観測問題、新解釈の全体像を示さなければ、「なぜ人間には$\mathcal{H}^+$の状態しか用意できないのか?」という疑問に答えることはできない。そのため、この疑問には、新解釈の全体像を示したのちに改めて考察することにしたい。
    \subsubsection{ラグランジュ表現での流体力学}
質点の動学と温度について説明を行ったので、次に連続体の流体力学の説明を行う。記載を簡便にするために、温度についての記載は省略することにする。新解釈では、時刻$t$における連続流体の状態(古典世界の状態)は、ある時間の推移とともに変化する宇宙の分割$\{V(t)_p\}$があって、
\begin{equation}
    \label{eq:classical_fluid_state}
    \bigotimes_p \ket{\rho_p(t),P_p(t)} \in \bigotimes_p \mathcal{H}_p^a
\end{equation}
とかけるとする。ただし、記載簡略化のためにケット$\ket{\cdot}_p$の添字$p$の記載は省略した。ボリューム間での物質の流出入はないものとする。ここで、$\rho_p(t)$はボリューム$V(t)_p$の密度(量子力学における密度行列、密度演算子ではなく、古典流体力学における密度である)、$P_p(t)$はボリューム$V(t)_p$の圧力である。
当然ながら、
\begin{equation}
    \label{eq:quantum_fluid_state_dynamics}
    \bigotimes_p \ket{\rho_p(t),P_p(t)} = e^{-iHt}\bigotimes_p \ket{\rho_p(0),P_p(0)}
\end{equation}
である。\par
新解釈では、$\mathcal{H}^i$との相互作用を無視すると、マクロ・古典的な物理量の固有状態は、時間が進んでも古典・マクロ的な物理量の固有状態であり続けるとしている(そのように$\mathcal{H}^i$を定義する)。従って、時刻$0$に密度、圧力について重ね合わせの状態でない(かつ異なるボリューム間でエンタングルしていない)\eqref{eq:classical_fluid_state}で記載できた状態は、時刻$t \neq 0$においても、\eqref{eq:classical_fluid_state}の形で記述でき、
\begin{equation}
    \label{eq:fluid_state_superposition}
   \int \cdots \int \prod_p \mathrm{d}\rho_p \; \varphi(\{\rho_p\}) \bigotimes_p \ket{\rho_p} \in \bigotimes_p \mathcal{H}_p^a
\end{equation}
と重ね合わせの状態(異なるボリューム間でエンタングルしていない状態)として記載する必要がある状態になることはないと考える(ただし、記載を簡略にするため、密度のみ記載し圧力は省略した)。ちなみに、\eqref{eq:classical_fluid_state}は、\eqref{eq:fluid_state_superposition}において、
\begin{equation}
   \varphi(\{\rho_p\}) = \prod_p \delta(\rho_p(t))
\end{equation}
とした状態である。\par
密度$\rho_p$及び圧力$P_p$の重ね合わせにならなければ、ボリューム$V(t)_p$については、その体積を十分に小さくした場合には、Lagrange的立場から流体力学の基礎方程式を導出する際の微小体積要素と同じ考えであるから、ボリューム$V(t)_p$の移動速度$\bm{V}_p$、$\nabla \cdot \bm{V}_p$等も計算することができ、\eqref{eq:quantum_fluid_state_dynamics}は、流体力学における運動方程式、すなわち、オイラー方程式、ナビエ–ストークス方程式等と一致すると新解釈では想定する(場の量子論QEDを解析的に解くことができれば、新解釈のこの想定が正しいかどうか確認することができるが、解析的に解くことはできていないため、新解釈ではそうなることを仮定する。科学的に正しいことを現時点で示すことはできないが、コペンハーゲン解釈における波束の収束よりは私にとって気持ち悪くないため、これを仮説として設定するのは波束の収束より妥当と私は感じる)。\par
例えば、最も単純な体積力を除いたオイラー方程式の場合、
\begin{equation}
    \frac{D{\bm{V}_p}}{Dt} = -\frac{1}{\rho_p } \mathrm{grad}\, P_p
\end{equation}
となる。ただし、$\mathrm{grad}\,P_p$は、近接しているボリュームから計算する必要が生じる。
    \subsubsection{オイラー表現での流体力学}
Euler表現の場合には、時刻によらない宇宙の分割$\{V_p\}$を用いて、
\begin{equation}
    \label{eq:euler_classical_fluid_state}
    \bigotimes_p \ket{\rho_p(t),P(t),\bm{V}_p(t)} \in \bigotimes_p \mathcal{H}_p^a
\end{equation}
とかけるとする。ボリューム間での物質の流出入が発生する。ここで、$\rho_p(t)$はボリューム$V(t)_p$の密度(量子力学における密度行列、密度演算子ではなく、古典流体力学における密度である)、$P_p(t)$はボリューム$V(t)_p$の圧力である。$\bm{V}_p(t)$はボリューム$V(t)_p$の群速度である。
当然ながら、
\begin{equation}
    \label{eq:quantum_euler_fluid_state_dynamics}
    \bigotimes_p \ket{\rho_p(t),P_p(t),\bm{V}_p(t)} = e^{-iHt}\bigotimes_p \ket{\rho_p(0),P_p(0),\bm{V}_p(0)}
\end{equation}
である。\par
新解釈では、$\mathcal{H}^i$との相互作用を無視すると、マクロ・古典的な物理量の固有状態は、時間が進んでも古典・マクロ的な物理量の固有状態であり続けるとしている(そのように$\mathcal{H}^t$と$\mathcal{H}^i$を定義する)。従って、時刻$0$に密度、圧力、群速度について重ね合わせの状態でない(かつ異なるボリューム間でエンタングルしていない)\eqref{eq:euler_classical_fluid_state}で記載できた状態は、時刻$t \neq 0$においても、\eqref{eq:euler_classical_fluid_state}の形で記述でき、
\begin{equation}
   \int \cdots \int \prod_p \mathrm{d}\rho_p
   \int \cdots \cdots \cdots \int \prod_p (\mathrm{d}V_x \mathrm{d}V_y \mathrm{d}V_z) \; \varphi(\{\rho_p\},\{\bm{V}_p\}) \bigotimes_p \ket{\rho_p,P_p,\bm{V}_p}
\end{equation}
と重ね合わせの状態(異なるボリューム間でエンタングルしている状態)として記載する必要がある状態になることはないと考える(ただし、記載を簡略にするため、圧力は省略した)。\par
密度$\rho_p$、圧力$\pi_p$及び群速度$\bm{V}_p(t)$の重ね合わせにならなければ、ボリューム$V(t)_p$については、その体積を十分に小さくした場合には、Euler的立場から流体力学の基礎方程式を導出する際の微小体積要素と同じ考えであるから、\eqref{eq:quantum_euler_fluid_state_dynamics}は、流体力学における運動方程式、すなわち、オイラー方程式、ナビエ–ストークス方程式等と一致すると新解釈では想定する(場の量子論QEDを解析的に解くことができれば、新解釈のこの想定が正しいかどうか確認することができるが、解析的に解くことはできていないため、新解釈ではそうなることを仮定する。科学的に正しいことを現時点で示すことはできないが、コペンハーゲン解釈における波束の収束よりは私にとって気持ち悪くないため、これを仮説として設定するのは波束の収束より妥当と私は感じる)。\par
例えば、最も単純な体積力を除いたオイラー方程式の場合、
\begin{equation}
    \frac{\partial{\bm{V}_p}}{\partial t} + (\bm{V}\cdot \nabla) \bm{V} = -\frac{1}{\rho_p } \mathrm{grad}\, P_p
\end{equation}
となる。ただし、$\mathrm{grad}\,P_p$は、近接しているボリュームから計算する必要が生じる。
    \subsubsection{電流と電荷}
電流、電荷については、電流が流れる導体そのものは動いていない場合で、電流についてEuler表現の場合には、時刻によらない宇宙の分割$\{V_p\}$を用いて、
\begin{equation}
    \label{eq:electric_current_state_euler}
    \bigotimes_p \ket{\rho^e_p(t),\bm{J}_p(t)} \in \bigotimes_p \mathcal{H}_p^a
\end{equation}
とかけるとする。ボリューム間での電荷の流出入が発生する。ここで、$\rho^e_p(t)$はボリューム$V(t)_p$の電荷密度、$\bm{J}_p(t)$はボリューム$V(t)_p$の電流密度である。
当然ながら、
\begin{equation}
    \bigotimes_p \ket{\rho^e_p(t),\bm{J}_p(t)} = e^{-iHt}\bigotimes_p \ket{\rho_p^e(0),\bm{J}_p(0)}
\end{equation}
である。\par
新解釈では、$\mathcal{H}^i$との相互作用を無視すると、マクロ・古典的な物理量の固有状態は、時間が進んでも古典・マクロ的な物理量の固有状態であり続けるとしている(そのように$\mathcal{H}^t$と$\mathcal{H}^i$を定義する)。従って、時刻$0$に電荷密度、電流密度について重ね合わせの状態でない(かつ異なるボリューム間でエンタングルしていない)状態は、時刻$t \neq 0$においても重なり合わせのない状態であり続けると考える。そのダイナミクスは、古典的な電磁気学に一致する。\par
電流が流れる導体そのものが動いている場合には、導体そのものについては、Lagrange表現を用いることとすると、時間の推移とともに変化する宇宙の分割$\{V(t)_p\}$があって、ボリューム間での物質全体の流出入はないが、電荷の流出入はあり、
\begin{equation}
    \label{eq:electric_current_state_lagrange}
    \bigotimes_p \ket{\rho_p(t),P_p(t),\rho^e_p(t),\bm{J}_p(t)} \in \bigotimes_p \mathcal{H}_p^a
\end{equation}
とかけるとする。脳細胞について考察するには、この表現が最も適していると考えられるため、この表現を今後基本的に用いる。
    \section{マクロ・古典的な状態の重ね合わせ}
新解釈では、$\mathcal{H}^i$との相互作用を無視すると、マクロ・古典的な物理量の固有状態は、時間が進んでも古典・マクロ的な物理量の固有状態であり続けると考えるが、$\mathcal{H}^i$との相互作用があると、マクロ・古典的な物理量の固有状態は、マクロ・古典的な物理量の固有状態の重ね合わせの状態になりえると考える。
    \subsection{離散的な重ね合わせの状態になった質点の運動}
時刻$0$において、位置 $n \in \{p\}$にのみ、質量$M$、速度$\bm{V}$で飛行している質点がある状況を考えよう。その状態は
\begin{equation}
    \label{initial_material_point_state}
    \ket{0}_1 \otimes \ket{0}_2 \otimes \ldots \otimes \ket{M,\bm{V}}_n \otimes \ket{0}_{n+1} \ldots \in \bigotimes_p \mathcal{H}_p^a
\end{equation}
とかける。ここで、$\ket{0}$は真空の状態であり、$\ket{M,\bm{V}}$は質量$M$、速度$\mathbf{V}$で特定される状態である。\par
質点以外に$\ket{\phi} \in \mathcal{H}^i$が存在し、全体では状態は
\begin{equation}
    \ket{0}_1 \otimes \ket{0}_2 \otimes \ldots \otimes \ket{M,\bm{V}}_n \otimes \ket{0}_{n+1} \ldots \otimes \ket{\phi} 
\end{equation}
とする。$\ket{\phi}$との相互作用により、時刻$dt$において、\eqref{initial_material_point_state}は、
\begin{equation}
\begin{aligned}
    \ket{0}_{\{V(dt)_p\},1} & \otimes \ket{0}_{\{V(dt)_p\},2} \otimes \ldots \\ 
    & \otimes (\alpha \ket{M,\bm{V}+d\bm{V}_1}_{\{V(dt)_p\},n} + \beta \ket{M,\bm{V}+d\bm{V}_2}_{\{V(dt)_p\},n}) \\
    & \qquad \otimes \ket{0}_{\{V(dt)_p\},n+1} \ldots 
\end{aligned}
\end{equation}
になるとする。ただし、Lagrange表現を用いており、時刻により異なる宇宙の分割$\{V(t)_p\}$を用いてヒルベルト空間を分割していることを明示するように記法を変更している。また、$\alpha,\beta$は$|\alpha|^2+|\beta|^2=1$を満たす複素数である。時刻$t>dt$では$\ket{\phi}$との相互作用はなくなると仮定すると、2つの状態 
\begin{equation}
\begin{aligned}
    & \ket{0}_{\{V(dt)_p\},1}  \otimes \ket{0}_{\{V(dt)_p\},2} \otimes \ldots \\ 
    & \qquad \otimes (\ket{M,\bm{V}+d\bm{V}_1}_{\{V(dt)_p\},n} \\
    & \qquad \qquad \otimes \ket{0}_{\{V(dt)_p\},n+1} \ldots ,
\end{aligned}
\end{equation}
\begin{equation}
\begin{aligned}
    &\ket{0}_{\{V(dt)_p\},1}  \otimes \ket{0}_{\{V(dt)_p\},2} \otimes \ldots \\ 
    & \qquad \otimes (\ket{M,\bm{V}+d\bm{V}_2}_{\{V(dt)_p\},n} \\
    & \qquad \qquad \otimes \ket{0}_{\{V(dt)_p\},n+1} \ldots 
\end{aligned}
\end{equation}
中の質点は等速運動をすると考えられ、新解釈では重ね合わせの状態になることはないので、任意の時刻$t>dt$における状態は、
\begin{equation}
\begin{aligned}
    &\alpha \ket{0}_{\{V_1(t)_p\},1}  \otimes \ket{0}_{\{V_1(t)_p\},2} \otimes \ldots \\ 
    & \qquad \otimes (\ket{M,\bm{V}+d\bm{V}_1}_{\{V_1(t)_p\},n} \\
    & \qquad \qquad \otimes \ket{0}_{\{V_1(t)_p\},n+1} \ldots \otimes \ket{\phi'} \\ 
    & + \beta \ket{0}_{\{V_2(t)_p\},1}  \otimes \ket{0}_{\{V_2(t)_p\},2} \otimes \ldots \\ 
    & \qquad \otimes (\ket{M,\bm{V}+d\bm{V}_2}_{\{V_2(t)_p\},n} \\
    & \qquad \qquad \otimes \ket{0}_{\{V_2(t)_p\},n+1} \ldots \otimes \ket{\phi''} 
\end{aligned}
\end{equation}
となる。このように、時刻により異なる宇宙の分割$\{V(t)_p\}$を用いるだけでなく、$\{V_1(t)_p\}$と$\{V_2(t)_p\}$という異なる宇宙のボリュームへの分割を用いて表現した状態の重ね合わせ状態を考える(フォーミュレーションを行う)ことが、新解釈の一つのポイントとである(無矛盾に数学として定義できるかどうかまでは私には考察できないが)。
    \subsection{異なる宇宙のボリュームへの分割の重ね合わせについて}
$\{V_1(t)_p\}$と$\{V_2(t)_p\}$という異なる宇宙のボリュームへの分割を用いて表現した状態の重ね合わせ状態を考えることは奇異に感じられるが(異なる分割ではなく、単一の分割でも奇異に感じられるかもしれないが)、これは、場の量子論を正準量子化やフォック空間として捉えるのではなく、場の波動関数として捉えれば、それほど奇異に感じられなくなると思われる。\par
ここでは、簡単に実スカラー場の場合のみを考えよう。そうすると、まず場$\phi$は、空間の点$x \in \mathbb{R}^3$から実数$\mathbb{R}$への写像$\mathbb{R}^3 \to \mathbb{R}$、すなわち、
\begin{equation}
    \phi : \mathbb{R}^3 \ni x \mapsto \phi (x) \in \mathbb{R}
\end{equation}
である。とりあえず二乗可積分の場合のみを考えることにして、$\phi \in L^2(\mathbb{R}^3)$としよう。次に、状態$\ket{\Psi}$は、写像$\phi$から複素振幅$\mathbb{C}$への写像(汎関数)、すなわち、
\begin{equation}
    \label{eq:taotal_universe}
    \ket{\Psi} :  L^2(\mathbb{R}^3) \ni \phi \mapsto \Psi (\phi) \in \mathbb{C}
\end{equation}
である。\par
ボリューム$V_p$の状態とは、ボリューム$V_p$に含まれる点の集合を同じ記号$V_p=\{x \in \mathbb{R}^3 | x \mathrm{\, is\, in\, the\, }V_p\}$で表すことにし、$V_p$上の実数値関数で積分可能なものの全体を$F(V_p)$で表すことにしたとき、写像$\Psi_p : F(V_p) \to \mathbb{C}$である。繰り返しになるが、
\begin{equation}
    \ket{\Psi_p}_{V_p} :  F(V_p) \ni \varphi_p \mapsto \Psi_p (\varphi_p) \in \mathbb{C}
\end{equation}
ということである。\par
宇宙を埋め尽くす空間において、$\Psi_p$が定義されていれば、そこから、\eqref{eq:taotal_universe}の$\Psi$を求めることができる。まず、$\varphi_p$を
\begin{equation}
     \varphi_p(x) = \phi(x),\; x \in V_p \subset \mathbb{R}^3
\end{equation}
と定義し、
\begin{equation}
    \label{eq:universe_wavefunction}
    \Psi (\phi) = \prod_p \Psi_p(\varphi_p)
\end{equation}
とすれば$\Psi (\phi)$が求まるからである。\par
このように、宇宙の分割の方法によらず、宇宙の状態は、$(L^2) = \{\Psi : L^2 \to \mathbb{C}\}$の元として表すことができる。$(L^2)$の元の和は、$(L^2)$に含まれるので、$\{V_1(t)_p\}$と$\{V_2(t)_p\}$という異なる宇宙のボリュームへの分割を用いて表現した状態の重ね合わせ状態も$(L^2)$に含まれる。\par
また、このことから、Euler表示とLagrange表示は相互に変換可能なことがわかるだろう。例えば、Lagrange表示の$\{\Psi_p\}$から\eqref{eq:universe_wavefunction}により$\Psi$を求め、今度はEuler表示の宇宙の分解で、\eqref{eq:universe_wavefunction}を満たす$\{\Psi'_p\}$を見つければ、それがEuler表示の状態になっていると考えられるからである。ただし、このEuler表示は、表示としては一意には定まらないことにも留意が必要である。例えば、$\Psi'_p$を$\alpha$倍して、$\Psi'_q$を$\alpha^{-1}$倍しても$\Psi$は変わらない。しかしこれは、量子力学において一般的にエンタングル状態の表示方法は一意に定まらないのと同じことであり、同じ状態を様々に表示できるということである。簡単に電子のスピンの状態で記載すると、
\begin{equation}
    \ket{\uparrow}=\frac{1}{\sqrt{2}}\ket{\leftarrow}+\frac{i}{\sqrt{2}}\ket{\rightarrow},
\end{equation}
\begin{equation}
    \ket{\downarrow}=\frac{1}{\sqrt{2}}\ket{\leftarrow}-\frac{i}{\sqrt{2}}\ket{\rightarrow}
\end{equation}
とすると、
\begin{equation}
    \frac{1}{\sqrt{2}}\ket{\uparrow} \otimes \ket{\downarrow} + \frac{1}{\sqrt{2}}\ket{\downarrow} \otimes \ket{\uparrow}
    = 
    \frac{1}{\sqrt{2}}\ket{\leftarrow} \otimes \ket{\rightarrow} + \frac{1}{\sqrt{2}}\ket{\rightarrow} \otimes \ket{\leftarrow}
\end{equation}
である。このように、表示方法には任意性がある。新解釈では、古典・マクロな物理量に関しては、この表示の任意性はないと考える。
    \subsection{離散的な重ね合わせの状態になった質点との衝突}
次に、重ね合わせの状態になった質点と重ね合わせの状態にない質点との衝突を考えよう。
時刻$0$において、位置 $n \in \{p\}$は、質量$M$、速度$\bm{V}_1$と速度$\bm{V}_2$の質点の重ね合わせ状態であり、位置$m \in \{p\}$は、質量$M$、速度$\bm{0}$(静止状態)の質点の状態であったとする。すなわち、
\begin{equation}
    \ket{0}_1 \otimes \ket{0}_2 \otimes \ldots \otimes \alpha \ket{M,\bm{V}_1}_n + \beta \ket{M,\bm{V}_2}_n \otimes \ket{0}_{n+1} \ldots \otimes \ket{M,\bm{0}}_m \otimes \ket{0}_{m+1} \ldots
\end{equation}
であったとする。位置$n$から速度$\bm{V}_1$で進むと、時刻$T$において、位置$m$に到達するとしよう。古典力学が教えるように、同一の質量の質点が衝突すると、動いていた方は停止し、静止していた方の質点は動いていた方の速度で動く。
Euler表現で記載するとこれは、
\begin{align}
    \label{eq:super_position_1}
    \ket{0}_1 & \otimes \ket{0}_2 \otimes \ldots \otimes \ket{M,\bm{V}_1}_{n_1(t)} \otimes \ket{0}_{n_1(t)+1} \ldots \otimes \ket{M,\bm{0}}_m \otimes \ket{0}_{m+1} \ldots \\
    &=e^{-iHt}\ket{0}_1 \otimes \ket{0}_2 \otimes \ldots \otimes \ket{M,\bm{V}_1}_n \otimes \ket{0}_{n+1} \ldots \otimes \ket{M,\bm{0}}_m \otimes \ket{0}_{m+1} \ldots \nonumber
\end{align}
ということである。ただし、ここで、$n_1(t)$は、位置$n$から速度$\bm{V}_1$で進んだ際の時刻$t$における位置である。\par
当然であるが、$\bm{V}_1$と異なる速度$\bm{V}_2$で進んでも位置$m$に到達することはない。従って、等速度で移動を続けることになり、
\begin{align}
    \label{eq:super_position_2}
    \ket{0}_1 & \otimes \ket{0}_2 \otimes \ldots \otimes \ket{M,\bm{V}_2}_{n_2(t)} \otimes \ket{0}_{n_2(t)+1} \ldots \otimes \ket{M,\bm{0}}_m \otimes \ket{0}_{m+1} \ldots \\
    &=e^{-iHt}\ket{0}_1 \otimes \ket{0}_2 \otimes \ldots \otimes \ket{M,\bm{V}_1}_n \otimes \ket{0}_{n+1} \ldots \otimes \ket{M,\bm{0}}_m \otimes \ket{0}_{m+1} \ldots \nonumber
\end{align}
が成り立つ。\par
\eqref{eq:super_position_1}を$\alpha$倍し、\eqref{eq:super_position_2}を$\beta$倍し、足し合わせると、
\begin{align}
    \label{eq:super_position_12}
    \ket{0}_1 & \ldots \otimes \alpha \ket{M,\bm{V}_1}_{n_1(t)} \ldots \otimes \ket{0}_{n_2(t)}  \ldots \otimes \ket{M,\bm{0}}_m \otimes\ldots \\
    & \quad + \ket{0}_1 \ldots \otimes \ket{0}_{n_1(t)} \ldots \otimes \beta \ket{M,\bm{V}_2}_{n_2(t)}  \ldots \otimes \ket{M,\bm{0}}_m \otimes \ldots \nonumber \\
    &=e^{-iHt}\ket{0}_1 \ldots \otimes \alpha \ket{M,\bm{V}_1}_n + \beta \ket{M,\bm{V}_2}_n \ldots \otimes \ket{M,\bm{0}}_m \otimes \ldots \nonumber
\end{align}
となる。\eqref{eq:super_position_12}の左辺を単純化して記載すると、
\begin{equation}
    \alpha \ket{M}_{n_1} \otimes \ket{0}_{n_2} + \beta \ket{0}_{n_1} \otimes \ket{M}_{n_2}
\end{equation}
という形をしており、これは、電子のスピンのエンタングル状態、すなわち、
\begin{equation}
    \label{eq:electron_spin_entangle}
    \alpha \ket{\uparrow} \otimes \ket{\downarrow} + \beta \ket{\downarrow} \otimes \ket{\uparrow}
\end{equation}
等と同様になっていることがわかる。マクロ・古典的な物理量の重ね合わせ状態は、時間の経過とともに、相互作用によりエンタングル状態となるのである。

    \subsection{離散的な重ね合わせの状態のコペンハーゲン解釈}
本稿は、コペンハーゲン解釈にはたたないが、実験結果においては、コペンハーゲン解釈は正しい予想を行うことができるとしている。コペンハーゲン解釈のもとで、\eqref{eq:super_position_12}に示される時刻$t$の状況で、質点の測定を行ったら、どのようになるだろうか。$n_1(t)$の位置を含み、$n_2(t)$も$m$も含まない範囲で質点を観測したら、その場合は、質点はあるか否かのいずれかであり、確定的に予想することはできない。質点がある結果を得た場合は、状態は
\begin{equation}
    \ket{0}_1 \ldots \otimes \ket{M,\bm{V}_1}_{n_1(t)} \ldots \otimes \ket{0}_{n_2(t)}  \ldots \otimes \ket{M,\bm{0}}_m \otimes\ldots 
\end{equation}
に収束し、質点がない結果を得た場合は
\begin{equation}
    \ket{0}_1 \ldots \otimes \ket{0}_{n_1(t)} \ldots \otimes \ket{M,\bm{V}_2}_{n_2(t)}  \ldots \otimes \ket{M,\bm{0}}_m \otimes \ldots \nonumber
\end{equation}
に状態は収束する。スピンの向きがエンタングルしている電子のペアの片方の電子のスピンの向きが測定で決定されれば、\eqref{eq:electron_spin_entangle}の状態が$\ket{\uparrow} \otimes \ket{\downarrow}$や$\ket{\downarrow} \otimes \ket{\uparrow}$に収束するように、マクロな質点の存在も、収束する(射影される)。\par
衝突する2つの質点の質量が異なる場合には、衝突前に片方の質点が停止していたとしても、衝突後は両方の質点が移動している。質量$M$と質量$M'$の質点が衝突したとすると、\eqref{eq:super_position_12}は、以下のように変わるだろう。
\begin{align}
    \label{eq:super_position_12_diffrent_mass}
    \ket{0}_1 & \ldots \otimes \alpha \ket{M,\bm{V}_1'}_{n'_1(t)} \ldots \otimes \ket{0}_{n_2(t)}  \ldots \otimes \ket{0}_m \ldots \otimes \ket{M',\bm{V}}_{m(t)} \otimes\ldots \\
    & \quad + \ket{0}_1 \ldots \otimes \ket{0}_{n'_1(t)} \ldots \otimes \beta \ket{M,\bm{V}_2}_{n_2(t)}  \ldots \otimes \ket{M',\bm{0}}_m \ldots \otimes \ket{0}_{m(t)} \otimes \ldots \nonumber \\
    &=e^{-iHt}\ket{0}_1 \ldots \otimes \alpha \ket{M,\bm{V}_1}_n + \beta \ket{M,\bm{V}_2}_n \ldots \otimes \ket{M',\bm{0}}_m \otimes \ldots .\nonumber
\end{align}
この状態を$n'_1(t)$の位置を含み、$n_2(t)$も$m$も$m(t)$も含まない範囲で質点を観測すれば、
\begin{equation}
    \ket{0}_1 \ldots \otimes \ket{M,\bm{V}_1'}_{n'_1(t)} \ldots \otimes \ket{0}_{n_2(t)}  \ldots \otimes \ket{0}_m \ldots \otimes \ket{M',\bm{V}}_{m(t)} \otimes\ldots
\end{equation}
か
\begin{equation}
    \ket{0}_1 \ldots \otimes \ket{0}_{n'_1(t)} \ldots \otimes \ket{M,\bm{V}_2}_{n_2(t)}  \ldots \otimes \ket{M',\bm{0}}_m \ldots \otimes \ket{0}_{m(t)} \otimes \ldots 
\end{equation}
に収束する(射影される)。観測していない方の質点の位置も収束し、異なる位置の重ね合わせ状態ではなく、確定した位置になっている。いずれの状態になっても、古典力学とは矛盾しない状態になっている。単純な例で示したのみであるが、マクロな質点が重なり合いの状態になったとしても、量子力学とも古典力学とも矛盾しないことがわかるだろう。マクロな状態の重ね合わせは、人間の直感には反するが、そうだとしても、量子力学(射影仮説を含む)が正しいとすると(加えて、新解釈で仮定している「$H^i$を除いたハミルトニアンによる時間発展のもとでは、マクロ・古典的な自由度(物理変数)の固有状態は、当該物理変数の固有状態であり続け、重ね合わせの状態にならない」が成り立てば)、特に矛盾が生じたりはしないのである。繰り返しになるが、これは、スピンの向きがエンタングルしている電子のペアの片方の電子のスピンの向きが測定で決定されれば、\eqref{eq:electron_spin_entangle}の状態が$\ket{\uparrow} \otimes \ket{\downarrow}$や$\ket{\downarrow} \otimes \ket{\uparrow}$に収束し、遥かに離れたもう一方の電子のスピンの向きも確定するのと同じである。
    \section{量子状態の随伴現象説}
新解釈においては、状態に随伴する意識・クオリアは、当該状態のマクロ・古典的な自由度(物理変数)にのみによって定まると考える。その前提となるのは、当該空間の状態の空間$\mathcal{H}$は、
\begin{equation}
    \mathcal{H} = \mathcal{H}^a \otimes  \mathcal{H}^t \otimes  \mathcal{H}^i
\end{equation}
とテンソル積に分解できるという考えである。ここで、$\mathcal{H}^a$は、マクロ・古典的自由度で指定される状態のヒルベルト空間であり、$\mathcal{H}^t$は熱平衡に達していると考えられるミクロな自由度で指定されるヒルベルト空間であり、$\mathcal{H}^i$は残りの自由度で指定される状態のヒルベルト空間である。
例えば、一人の人を含む範囲の空間の状態を$\ket{A}$ととし、その人に意識・クオリア$|a \rfloor$が随伴しているとする。すなわち、$\ket{A} \mapsto |a \rfloor$とする。
    \printbibliography
\end{document}