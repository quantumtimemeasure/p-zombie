\subsection{離散的な重ね合わせの状態のコペンハーゲン解釈}
本稿は、コペンハーゲン解釈にはたたないが、実験結果においては、コペンハーゲン解釈は正しい予想を行うことができるとしている。コペンハーゲン解釈のもとで、\eqref{eq:super_position_12}に示される時刻$t$の状況で、質点の測定を行ったら、どのようになるだろうか。$n_1(t)$の位置を含み、$n_2(t)$も$m$も含まない範囲で質点を観測したら、その場合は、質点はあるか否かのいずれかであり、確定的に予想することはできない。質点がある結果を得た場合は、状態は
\begin{equation}
    \ket{0}_1 & \ldots \otimes \ket{M,\bm{V_1}}_{n_1(t)} \ldots \otimes \ket{0}_{n_2(t)}  \ldots \otimes \ket{M,\bm{0}}_m \otimes\ldots 
\end{equation}
に収束し、質点がない結果を得た場合は
\begin{equation}
    \ket{0}_1 \ldots \otimes \ket{0}_{n_1(t)} \ldots \otimes \ket{M,\bm{V_2}}_{n_2(t)}  \ldots \otimes \ket{M,\bm{0}}_m \otimes \ldots \nonumber
\end{equation}
に状態は収束する。スピンの向きがエンタングルしている電子のペアの片方の電子のスピンの向きが測定で決定されれば、\eqref{eq:electron_spin_entangle}の状態が$\ket{\uparrow} \otimes \ket{\downarrow}$や$\ket{\downarrow} \otimes \ket{\uparrow}$に収束するように、マクロな質点の存在も、収束する(射影される)。