\subsection{1.2 随伴現象説のフォーミュレーション}
前述のように、物理的な状態は、意識・クオリアが随伴する状態と、随伴しない状態がある。しかも、意識・クオリアが随伴するかどうかは、物理的な状態によって定まっている。したがって、意識・クオリアが随伴する状態すべての集合 $\mathfrak{C}$ を定義することができる。すなわち、$|A\rangle\in\mathfrak{C}$であれば、状態$|A\rangle$は意識・クオリアを随伴している状態であり、$|B\rangle\notin\mathfrak{C}$であれば、状態$|B\rangle$は意識・クオリアを随伴していない状態である。\\
 その一部に意識・クオリアが随伴する状態を含んでいれば、意識・クオリアがある状態と考えられるため(必ずしもそうかんがえる本稿では、脳に意識・クオリアがあれば人間には意識・クオリアがあるという立場をとるため)、$|A\rangle = |C\rangle\otimes |d\rangle$とテンソル積に状態が分解でき、$|C\rangle\in\mathfrak{C}$であれば、