\subsection{1.2 随伴現象説のフォームレーション}
 前述のように、物理的な状態は、意識・クオリアが随伴する状態と、随伴しない状態がある。しかも、意識・クオリアが随伴するかどうかは、物理的な状態によって定まっている。したがって、意識・クオリアが随伴する状態すべての集合 $\mathfrak{C}$ を定義することができる。すなわち、$|A\rangle\in\mathfrak{C}$であれば、状態$|A\rangle$は意識・クオリアを随伴している状態であり、$|B\rangle\notin\mathfrak{C}$であれば、状態$|B\rangle$は意識・クオリアを随伴していない状態である。
 意識を随伴している状態を含んでいれば、