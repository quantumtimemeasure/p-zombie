\subsubsection{相互作用表示における不連続性}
相互作用表示においては、時間発展に伴い、意識・クオリアに関わる状態を指定する物理量、例えば$\{E_{i,p}\}$は、非連続に変化することに留意が必要である。これは、物理的な状態が非連続に変化するということではない。$others$を含めた状態$|\{ E_{i,p}\},others \rangle$は連続的に変化する。しかし、相互作用表示は、各$C_j$の代表点$c_j \in C_j$をとって、物理的な状態を$|c_j,others \rangle$のように書きたいという要請を別の方法で実現したものであるため、$others$を無視して$\{E_{i,p}\}$のみを見た場合には、不連続に変化するのである。\\
 これは、実数$x \in \mathbb{R}$を整数$n \in \mathbb{Z}$とそこからの差分$x-n = y \in [0,1)$の組$(n,y)$として表現した場合、$x$の連続的な変化に対して、$n$が不連続に(実数の意味で不連続に)変化するようなものである。$n$が$\{E_{i,p}\}$に、$y$が$others$に相当する。$x = (n,y)$に非連続性がないように、状態$|\{ E_{i,p}\},others \rangle$に非連続性はない。
