\subsubsection{電流と電荷}
でんりゅEuler表現の場合には、時刻によらない宇宙の分割$\{V_p\}$を用いて、
\begin{equation}
    \label{eq:euler_classical_fluid_state}
    \bigotimes_p \ket{\rho_p(t),\bm{J}_p(t)} \in \bigotimes_p \mathcal{H}_p^a
\end{equation}
とかけるとする。ボリューム間での物質の流出入が発生する。ここで、$\rho_p(t)$はボリューム$V(t)_p$の電荷密度、$\bm{J}_p(t)$はボリューム$V(t)_p$の電流密度である。
当然ながら、
\begin{equation}
    \bigotimes_p \ket{\rho_p(t),\bm{J}_p(t)} = e^{-iHt}\bigotimes_p \ket{\rho_p(0),\bm{J}_p(0)}
\end{equation}
である。\par
新解釈では、$\mathcal{H}^i$との相互作用を無視すると、マクロ・古典的な物理量の固有状態は、時間が進んでも古典・マクロ的な物理量の固有状態であり続けるとしている(そのように$\mathcal{H}^t$と$\mathcal{H}^i$を定義する)。従って、時刻$0$に電荷密度、電流密度について重ね合わせの状態でない(かつ異なるボリューム間でエンタングルしていない)状態は、時刻$t \neq 0$においても重なり合わせのない状態であり続けると考える。そのダイナミクスは、古典的な電磁気学に一致する。