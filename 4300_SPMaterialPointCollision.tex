\subsection{離散的な重ね合わせの状態になった質点との衝突}
次に、重ね合わせの状態になった質点と重ね合わせの状態にない質点との衝突を考えよう。
時刻$0$において、位置 $n \in \{p\}$は、質量$M$、速度$\bm{V_1}$と速度$\bm{V_2}$の質点の重ね合わせ状態であり、位置$m \in \{p\}$は、質量$M$、速度$\bm{0}$(静止状態)の質点の状態であったとする。すなわち、
\begin{equation}
    \ket{0}_1 \otimes \ket{0}_2 \otimes \ldots \otimes \alpha \ket{M,\bm{V_1}}_n + \beta \ket{M,\bm{V_2}}_n \otimes \ket{0}_{n+1} \ldots \otimes \ket{M,\bm{0}}_m \otimes \ket{0}_{m+1} \ldots
\end{equation}
であったとする。位置$n$から速度$\bm{V_1}$で進むと、時刻$T$において、位置$m$に到達するとしよう。古典力学が教えるように、同一の質量の質点が衝突すると、動いていた方は停止し、静止していた方の質点は動いていた方の速度で動く。
Euler表現で記載するとこれは、
\begin{align}
    \ket{0}_1 & \otimes \ket{0}_2 \otimes \ldots \otimes \ket{M,\bm{V_1}}_{n_1(t)} \otimes \ket{0}_{n_1(t)+1} \ldots \otimes \ket{M,\bm{0}}_m \otimes \ket{0}_{m+1} \ldots \\
    &=e^{-iHt}\ket{0}_1 \otimes \ket{0}_2 \otimes \ldots \otimes \ket{M,\bm{V_1}}_n \otimes \ket{0}_{n+1} \ldots \otimes \ket{M,\bm{0}}_m \otimes \ket{0}_{m+1} \ldots \nonumber
\end{align}
ということである。ただし、ここで、$n_1(t)$は、位置$n$から速度$\bm{V_1}$で進んだ際の時刻$t$における位置である。\par
当然であるが、$\bm{V_1}$と異なる速度$\bm{V_2}$で進んでも位置$m$に到達することはない。