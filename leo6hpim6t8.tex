\subsubsection{古典極限における質点運動}
最もシンプルなケースとして、位置 $n \in \{p\}$にのみ、質量$M$、速度$V_k$、温度$T$で飛行している質点がある状況を考えよう。その状態は
\begin{equation}
    |0\rangle_1 \otimes |0\rangle_2 \otimes \ldots \otimes |M,V_1,V_2,V_3,T\rangle_n \otimes |0\rangle_{n+1} \ldots \in \bigotimes_p \mathcal{H}_p^a
\end{equation}
とかける。ここで、$|0\rangle$は真空の状態であり、$|M,V_1,V_2,V_3,T\rangle$は質量$M$、速度$V_k$、温度$T$で特定される状態である。\\
 上記を時刻$0$における状態とし、その時点では位置$n$は原点にあったとする。時刻$t$においては、位置$n$が$V_t$にある$V^t_p$を用いてヒルベルト空間を分割して、
\begin{equation}
    \mathcal{H} = \bigotimes_p \mathcal{H}^t_p^a,
\end{equation}