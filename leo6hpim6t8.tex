\subsubsection{古典極限における質点運動}
最もシンプルなケースとして、位置 $n \in \{p\}$にのみ、質量$M$、速度$\mathbf{V}$、逆温度$\beta$で飛行している質点がある状況を考えよう。その状態は
\begin{equation}
    |0\rangle_1 \otimes |0\rangle_2 \otimes \ldots \otimes |M,\mathbf{V},\beta \rangle_n \otimes |0\rangle_{n+1} \ldots \in \bigotimes_p \mathcal{H}_p^a
\end{equation}
とかける。ここで、$|0\rangle$は真空の状態であり、$|M,\mathbf{V},\beta \rangle$は質量$M$、速度$\mathbf{V}$、逆温度$\beta$で特定される状態である。\\
 上記を時刻$0$における状態とし、その時点では位置$n$は原点にあったとする。時刻$t$においては、位置$n$が$\mathbf{V}t$にある宇宙の分割$\{V(t)_p\}$を用いてヒルベルト空間を分割して、
\begin{equation}
    \mathcal{H} = \bigotimes_p \mathcal{H}^a(t)_p
\end{equation}
とすると、時刻$t$の状態は同じく、
\begin{equation}
    |0\rangle_{(t)1} \otimes |0\rangle_{(t)2} \otimes \ldots \otimes |M,\mathbf{V},\beta \rangle_{(t)n} \otimes |0\rangle_{(t)n+1} \ldots \in \bigotimes_p \mathcal{H}^a(t)_p
\end{equation}
とかける。そして、それはハミルトニアン$H$による時間発展であるから、
\begin{equation}
\begin{aligned}
    e^{-iHt}|0\rangle_1 \otimes |0\rangle_2 \otimes \ldots \otimes |M,\mathbf{V},\beta \rangle_n \otimes |0\rangle_{n+1} \ldots \\
    =  |0\rangle_{(t)1} \otimes |0\rangle_{(t)2} \otimes \ldots \otimes |M,\mathbf{V},\beta \rangle_{(t)n} \otimes |0\rangle_{(t)n+1} \ldots
\end{aligned}
\end{equation}
である。\\
 時刻により異なる宇宙の分割$\{V(t)_p\}$を用いず、同じ分割$\{V_p\}$を用いるならば、
\begin{equation}
\begin{aligned}
    e^{-iHt}|0\rangle_1 \otimes |0\rangle_2 \otimes \ldots \otimes |M,\mathbf{V},\beta \rangle_{n(0)} \otimes |0\rangle_{n+1} \ldots \\
    =  |0\rangle_{1} \otimes |0\rangle_{2} \otimes \ldots \otimes |M,\mathbf{V},\beta \rangle_{n(t)} \otimes |0\rangle_{n(t)+1} \ldots
\end{aligned}
\end{equation}
となる。ただしここで、$V_{n(0)}$は原点を含むボリュームであり、$V_{n(t)}$は、が$\mathbf{V}t$の位置を含むボリュームである。\\
 質量$M$、速度$\mathbf{V}$、逆温度$\beta$の質点と質量$M'$、速度$\mathbf{V}'$、逆温度$\beta$の質点が衝突し、それぞれの質点の速度が速度$\mathbf{V}''$及び$\mathbf{V}'''$になったとすると、
\begin{equation}
\begin{aligned}
    e^{-iHt}|0\rangle_1 \otimes |0\rangle_2 \otimes \ldots \otimes |M,\mathbf{V},\beta \rangle_{n(0)} \otimes |0\rangle_{n+1} \ldots \otimes |M',\mathbf{V}',\beta \rangle_{m(0)} \otimes |0\rangle_{m+1} \ldots \\
    =  |0\rangle_{1} \otimes |0\rangle_{2} \otimes \ldots \otimes |M,\mathbf{V}'',\beta \rangle_{n(t)} \otimes |0\rangle_{n(t)+1} \ldots
    \otimes |M',\mathbf{V}',\beta \rangle_{m(0)} \otimes |0\rangle_{m+1} \ldots \end{aligned}
\end{equation} 