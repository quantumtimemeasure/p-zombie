\subsubsection{古典極限における質点運動}
最もシンプルなケースとして、位置$n \$にのみ、質量$M$、速度$V$、温度$T$で飛行している質点がある状況を考えよう。その状態は
\begin{equation}
    |0\rangle_{p_1} \otimes |0\rangle_{p_2} \otimes \ldots \otimes |M,V,T\rangle_{p_n} \otimes |0\rangle_{p_{n+1}} \ldots \in \bigotimes_p \mathcal{H}_p^a
\end{equation}
とかける。ここで、$|0\rangle$は真空の状態であり、$|M,V,T\rangle$は質量$M$、速度$V$、温度$T$で特定される状態である。