\section{マクロ・古典的な状態}
本性では、マクロ・古典的な状態の定義を行う。また、本章では、意識・クオリアは取り扱わないため、特に記載しない場合、シュレディンガー表示を用いる。
\subsection{マクロに静止している場合の分割}
宇宙を分子の大きさより十分に大きく、しかし、ほぼ熱平衡に達していると考えられる程度には十分に小さいボリュームの集まり$\{V_p\}$に分割し、宇宙の状態のヒルベルト空間$\mathcal{H}$を$V_p$の状態のヒルベルト空間$\mathcal{H}_p$ のテンソル積に分割する(分子の大きさよりも十分に大きいとする必要はない可能性も高いが(いや、むしろ分子の大きさよりも小さくても以下の記載が成り立つとする必要性が高いが)、私も分子の粒子描像に毒されているので、ここではいったん分子の大きさよりも大きいとする)。すなわち、
\begin{equation}
    \mathcal{H} = \bigotimes_p \mathcal{H}_p
\end{equation}
とする(正確には、$\bigotimes_p \mathcal{H}_p$の元のほとんどは$\mathcal{H}$には含まれないため、分割とは異なるが、簡単のために分割の表現を用いる。フォック空間も、すべてが物理的にありえる状態ではなく、対象又は反対象な元のみが物理的な状態であるので、それと同様に$\bigotimes_p \mathcal{H}_p$の元のうち一部のみが物理的な状態に対応する)。\par
次に、各$\mathcal{H}_p$をマクロ・古典的とミクロのテンソル積へ分割することを考える。すなわち、マクロ・古典的変数で指定される状態のヒルベルト空間$\mathcal{H}_p^a$と残りの自由度のうち熱平衡に達していると考えられる自由度で指定されるヒルベルト空間$\mathcal{H}_p^t$と残りの自由度で指定される状態のヒルベルト空間$\mathcal{H}_p^i$のテンソル積に分割する($\mathcal{H}_p^t$の意味が不明と思われるが、説明が長くなるため、後で説明する)。
\begin{equation}
    \mathcal{H}_p = \mathcal{H}_p^a \otimes \mathcal{H}_p^t \otimes \mathcal{H}_p^i
\end{equation}
となる。ここで、
\begin{equation}
    \mathcal{H}^i = \bigotimes_p \mathcal{H}_p^i,
\end{equation}
\begin{equation}
    \mathcal{H}^t = \bigotimes_p \mathcal{H}_p^t
\end{equation}
を導入すると、
\begin{equation}
    \mathcal{H} = \bigotimes_p \mathcal{H}_p^a \otimes  \mathcal{H}^t \otimes  \mathcal{H}^i
\end{equation}
となる。\par
ハミルトニアン$H$は、以下のように分割されると考える。
\begin{equation}
H=\sum_pH_p^a + H^t + H^i + \sum_p H_p^{ai\,int} + \sum_{p \neq q} H^{int}_{p,q} + \sum_{p } H_p^{at\,int}
\end{equation}
それぞれの意味は、記号から想像されるとおりであり、一部説明すると、$\sum_p H_p^{ai\,int}$は熱平衡にないミクロな状態とマクロ・古典的な状態の相互作用である。例えば、空気中を飛行する電子と電子の測定器との相互作用がこれに該当する。空気中を飛行する電子は、マクロな状態である空気と熱平衡の状態にはないと考えられる。\par
$\sum_{p \neq q}H^{int}_{p,q}$は異なる位置のマクロ・古典的な自由度(物理変数)間の相互作用であり、古典物理学における通常の相互作用である(ただし、主たる内容は異なる位置への流入、流出であり、いわゆる相互作用ではないことに注意が必要である)。そのため、本来は$\sum_{p \neq q}H^{a \, int}_{p,q}$と記載すべきと考えられるが、$a$の記載を省略したものである。本稿では、相対論的量子力学を考えているので、近接相互作用しかなく、異なるボリューム間の相互作用は、表面での相互作用に限られるとするが、非相対論的な量子力学を考えることもできる。\par
$\sum_{p \neq q} H_{p,q}^{t\,int}$の項が明示的に無いのは、熱平衡にある状態の相互作用は、ランダムさがあり、それを$H^t=\sum_pH_p^t$で考慮するならば、ボリューム間の相互作用は、マクロ・古典的な相互作用に集約できると考えられるためである($H^t=\sum_pH_p^t+\sum_{p \neq q} H_{p,q}^{t\,int}$としても良いが)。マクロ・古典的な自由度には、温度も含まれ(エネルギーを通して間接的に含まれると考えることもできる)、異なる温度のボリュームの接触に基づく温度の均一化は、$\sum_{p \neq q} H^{int}_{p,q}$の方に含まれる。そのため、$\sum_{p \neq q} H_{p,q}^{t\,int}$の項目は不要としてよいと見込まれる(後述する)。\par
$\sum_{p } H_p^{at\,int}$は、$\mathcal{H}_p^a$と$\mathcal{H}^t$の間の相互作用、すなわち、熱平衡にあるミクロな状態とマクロ・古典的な状態の間の相互作用である。この相互作用は、例えば水中の微粒子にブラウン運動をもたらすものと考えられるが、その位置づけは従来のブラウン運動の解釈とは大きく異なるため、その説明は後述することとしたい。