\subsection{随伴現象説のフォーミュレーション}
\subsubsection{基本}
前述のように、物理的な状態は、意識・クオリアが随伴する状態と、随伴しない状態がある。しかも、意識・クオリアが随伴するかどうかは、物理的な状態によって定まっている。したがって、意識・クオリアが随伴する状態すべての集合 $\mathfrak{C}$ を定義することができる。すなわち、$|A\rangle \in \mathfrak{C}$であれば、状態$|A\rangle$は意識・クオリアを随伴している状態であり、$|B\rangle \notin \mathfrak{C}$であれば、状態$|B\rangle$は意識・クオリアを随伴していない状態である(前述したように、新解釈では、マクロな状態においても重ね合わせの)。\\
 その一部に意識・クオリアが随伴する状態を含んでいれば、意識・クオリアがある状態と考えられるため(必ずしもそう考えるのが必須ではないが、本稿では、脳に意識・クオリアがあれば人間には意識・クオリアがあるという立場をとるため)、$|A\rangle = |C\rangle \otimes |d\rangle$とテンソル積に状態が分解でき、$|C\rangle\in\mathfrak{C}$であれば、$|A\rangle\in\mathfrak{C}$と本稿ではする。\\
 $|C\rangle\in\mathfrak{C}$であれば、$|A\rangle\in\mathfrak{C}$とする背景には、脳が冗長性、可塑性を有していることがある。すなわち、テンソル積への分解の意味で、最小の意識・クオリアがある状態を定義することが難しい。例えば、$|A\rangle = |c\rangle \otimes |d\rangle \otimes |e\rangle$とテンソル積に状態が分解でき、$|c\rangle \otimes |d\rangle$も$|c\rangle \otimes |e\rangle$も意識・クオリアを随伴する状態であるが、$|c\rangle$のみでは意識・クオリアを随伴しない場合、意識・クオリを随伴する最小の状態を定義することが難しい。