\section{量子状態の随伴現象説}
新解釈においては、状態に随伴する意識・クオリアは、当該状態のマクロ・古典的な自由度(物理変数)にのみによって定まると考える。その前提となるのは、当該空間の状態の空間$\mathcal{H}$は、
\begin{equation}
    \mathcal{H} = \mathcal{H}^a \otimes  \mathcal{H}^t \otimes  \mathcal{H}^i
\end{equation}
とテンソル積に分解できるという考えである。ここで、$\mathcal{H}^a$は、マクロ・古典的自由度で指定される状態のヒルベルト空間であり、$\mathcal{H}^t$は熱平衡に達していると考えられるミクロな自由度で指定されるヒルベルト空間であり、$\mathcal{H}^i$は残りの自由度で指定される状態のヒルベルト空間である。
例えば、一人の人を含む範囲の空間の状態を$\ket{A}$ととし、その人に意識・クオリア$|a \rfloor$が随伴しているとする。すなわち、$\ket{A} \mapsto |a \rfloor$とする。