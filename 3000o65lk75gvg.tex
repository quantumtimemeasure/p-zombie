\section{マクロ・古典的な状態}
本性では、マクロ・古典的な状態の定義を行う。また、本性では特に記載しない場合、シュレディンガー表示を用いる。
\subsection{マクロに静止している場合の分割}
宇宙を分子の大きさより十分に大きく、しかし、ほぼ熱平衡に達していると考えられる程度には十分に小さいボリューム$V_p$に分割し、宇宙の状態のヒルベルト空間$\mathcal{H}$を$V_p$の状態のヒルベルト空間$\mathcal{H}_p$ のテンソル積に分割する(これは、原子、分子といった粒子的世界観に)。すなわち、
\begin{equation}
    \mathcal{H} = \bigotimes_p \mathcal{H}_p
\end{equation}
とする。\\
 次に、各$\mathcal{H}_p$をマクロとミクロのテンソル積へ分割することを考える。すなわち、マクロ変数で指定される状態のヒルベルト空間$\mathcal{H}_p^a$と残りの自由度のうち熱平衡に達していると考えられる自由度で指定されるヒルベルト空間$\mathcal{H}_p^t$と残りの自由度で指定される状態のヒルベルト空間$\mathcal{H}_p^i$のテンソル積に分割する($\mathcal{H}_p^t$の意味が不明と思われるが、説明が長くなるため、後で説明する)。
\begin{equation}
    \mathcal{H}_p = \mathcal{H}_p^a \otimes \mathcal{H}_p^t \otimes \mathcal{H}_p^i
\end{equation}
となる。ここで、
\begin{equation}
    \mathcal{H}^i = \bigotimes_p \mathcal{H}_p^i,
\end{equation}
\begin{equation}
    \mathcal{H}^t = \bigotimes_p \mathcal{H}_p^t
\end{equation}
を導入すると、
\begin{equation}
    \mathcal{H} = \bigotimes_p \mathcal{H}_p^a \otimes  \mathcal{H}_p^t \otimes  \mathcal{H}_p^i
\end{equation}
となる。\\
 ハミルトニアン$H$は、以下のように分割されると考える。
\begin{equation}
H=\sum_pH_p^a + H^t + H^i + \sum_p H_p^i + \sum_{p \neq q} H^{int}_{p,q}
\end{equation}
それぞれの意味は、記号から想像されるとおりであり、一部説明すると、$\sum_p H_p^i$は熱平衡にないミクロな状態とマクロな状態の相互作用である。例えば、空気中を飛行する電子と電子の測定器との相互作用がこれに該当する。空気中を飛行する電子は、マクロな状態である空気と熱平衡の状態にはないと考えられる。$\sum_{p \neq q} H^{int}_{p,q}$は異なる位置のマクロな自由度(物理変数)間の相互作用であり、古典物理学における通常の相互作用である。本稿では、相対論的量子力学を考えているので、近接相互作用しかなく、異なるボリューム間の相互作用は、表面での相互作用に限られるとするが、非相対論的な量子力学を考えることもできる。\\
 $\sum_p H_p^t$の項が無いのは、熱平衡にあるボリューム間の相互作用は、マクロな相互作用に集約できると考えられるためである。マクロな自由度には、温度も含まれ(エネルギーを通して間接的に含まれると考えることもできる)、異なる温度のボリュームの接触に基づく温度の均一化は、$\sum_{p \neq q} H^{int}_{p,q}$の方に含まれる。そのため、$\sum_p H_p^t$の項目は不要としてよいと見込まれる。