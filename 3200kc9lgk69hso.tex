\subsection{古典極限における流体力学}
前節のフォミュレーションにおいては、古典極限は、以下の2式となる。
\begin{equation}
    \mathcal{H} = \bigotimes_p \mathcal{H}_p^a,
\end{equation}
\begin{equation}
H=\sum_pH_p^a+ \sum_{p \neq q} H^{int}_{p,q}.
\end{equation}
 古典極限の意味は、上記のハミルトニアン$H$による時間発展のもとでは、マクロ・古典的な自由度(物理変数)の固有状態は、当該物理変数の固有状態でありつつ