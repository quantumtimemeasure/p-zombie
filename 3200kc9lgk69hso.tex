\subsection{古典世界}
前節のフォーミュレーションにおいては、いわゆる古典極限($\hbar$を$0$とした極限)は、以下の2式となる。
\begin{equation}
    \mathcal{H} = \bigotimes_p \mathcal{H}_p^a,
\end{equation}
\begin{equation}
H=\sum_pH_p^a+ \sum_{p \neq q} H^{int}_{p,q}.
\end{equation}
 新解釈では、これを古典世界と呼ぶことにする。極限をとったわけではなく、$\bigotimes_p \mathcal{H}_p^a \otimes  \mathcal{H}_p^t \otimes  \mathcal{H}_p^i$から$\bigotimes_p \mathcal{H}_p^a$だけを抜き出したものだからである(ちなみに、$\mathcal{H}_p^t \otimes  \mathcal{H}_p^i$の影響が小さいから無視したと考えれば、極限をとったことになる。新解釈のポイントは、影響が小さいから無視できるのでなく、それらの影響が大きくても、古典世界を分割して取り出すことができると考えるところにある)。これを世界全体から古典世界を抜き出したと表現することにしよう。古典世界のポイントは、上記のハミルトニアン$H$による時間発展のもとでは、マクロ・古典的な自由度(物理変数)の固有状態は、当該物理変数の固有状態であり続け、重ね合わせの状態にならないということである(これは近似的にそうなだけである可能性がある)。古典世界の状態は重ね合わせの状態にならないという