\section*{はじめに}
本稿では、コペンハーゲン解釈、多世界解釈に代わる新しい量子力学の解釈を示そうと思う。新解釈に名前はまだない。新解釈は、波束の収束・射影仮説が不要な点で、多世界解釈に近い解釈であるが、遠い銀河を含めて世界が分かれるとは考えないことから、いわゆる多世界解釈とは異なる。EverettのもとのRelative-State Formulationにより近いものと考えられる。\par
本稿の検討は、「哲学的ゾンビ」のタイトルが示すように、哲学・形而上学に属するものであり、自然科学に属するものではない。本稿は、実験結果に関する限り、コペンハーゲン解釈に基づく量子力学(場の理論を含む)が正しいことを前提としている。そのため、新解釈によって、実験結果について新しい予言ができるようになることはない。異なる考え方があり、そのいずれが正しいか実験で確認しようがないとき、いずれの考え方を選ぶか考察することは、科学ではなく、哲学と私は考えている。したがって、本稿の考察は哲学である。自然科学の理論としては、コペンハーゲン解釈と新解釈・本稿の間に違いはなく、同じものである。\par
ただし、マクロな巨視的な状態に対しても、重ね合わせの原理が成り立つと考える。重ね合わせの原理が成り立つミクロな状態と重ね合わせのないマクロな状態の境界は知られていないため、これはコペンハーゲン解釈と矛盾するものでない。\par
本稿の構成(予定)は以下のとおりである。まず、新解釈は、随伴現象説に基づくものであるため、第1章で、まずその説明から始める。次に、古典的な世界がどのようにして量子力学、ヒルベルト空間から生み出されるのかの仮説を第2章及び第3章で述べる。そして、第4章で、その古典的な世界と随伴現象説の関係を述べる。最後に第5章で、存在とは何かについての、私見を科学的実在論を中心に述べる。また、本稿はギリシア時代の原子論と同様に科学的な根拠がほぼ存在しない考察であるが、それが科学的な根拠を持つものになる可能性について整理して述べる。