\subsubsection{1.2.2 意識・クオリアにかかる相互作用表示}
記載を勘弁にするために、意識・クオリアにかかる相互作用表示を導入する。意識・クオリアは、脳細胞の物理的化学的電気的反応であり、静的なものではない。人間の脳がどのように意識・クオリアをエンコードしているかは解明されていないが、脳が外界の情報をエンコードするためにシンプルなものとして、Rate coding、Temporal coding、Population codingがあることが知られている\cite{wikipediac}。これらはすべて、ダイナミックなものである。しがって同一のクオリアが継続している間の状態であっても、シュレディンガー表示のもとでは、時刻により状態が異なってしまう。そのため、同一の意識・クオリアが継続している間同じ状態で表すことができるよう相互作用表示をもちいる(正確には、そもそも、ダイナミックな変化に意識・クオリアは随伴しているのであり、特定の時刻の状態に意識・クオリアが随伴するわけではない)。

