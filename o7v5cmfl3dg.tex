\subsubsection{1.2.2 意識・クオリアにかかる相互作用表示}
記載を勘弁にするために、意識・クオリアにかかる相互作用表示を導入する。意識・クオリアは、脳細胞の物理的化学的電気的反応であり、静的なものではない。人間の脳がどのように意識・クオリアをエンコードしているかは解明されていないが、脳が外界の情報をエンコードするためにシンプルなものとして、Rate coding、Temporal coding、Population codingがあることが知られている\cite{wikipediac}。これらはすべて、ダイナミックなものである。しがって同一のクオリアが継続している間の状態であっても、シュレディンガー表示のもとでは、時刻により状態が異なってしまう。そのため、同一の意識・クオリアが継続している間同じ状態で表すことができるよう相互作用表示を用いることとする(正確には、そもそも、ダイナミックな変化に意識・クオリアは随伴しているのであり、特定の時刻の状態に意識・クオリアが随伴するわけではない)。\\
 すなわち、ハミルトニアン$H$を、意識・クオリアの随伴にとって内在的な$H_c$とその他の部分$H_o$に下記のように分解し、
 \begin{equation}
  H=H_c + H_o
\end{equation}
シュレディンガー表示の状態$|S(t)\rangle$と
 \begin{equation}
  |I(t)\rangle = e^{iHt}|S(t)\rangle
\end{equation}
の関係にある相互作用表示の$|I(t)\rangle$を本稿では基本的に用いる。この表示であれば、時刻によらず同一の意識・クオリアが継続している間は、同じ状態で表すことができる。\\
 若干単純化して(たぶん脳科学の観点からは誤って)、視覚の場合について表現するならば、$H_c$は脳と視覚細胞の相互作用のハミルトニアン、$H_o$は視覚細胞と光子の間の相互作用とかんがえれ

