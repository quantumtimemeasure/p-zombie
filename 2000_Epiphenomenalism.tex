\section{随伴現象説}
随伴現象説とは、
\begin{quotation}
心の哲学において、物質と意識の間の因果関係について述べた形而上学的な立場のひとつで、『意識やクオリアは物質の物理的状態に付随しているだけの現象にすぎず、物質にたいして何の因果的作用ももたらさない』というもの。\\物質と意識を別の存在であると捉える二元論の立場を取りつつ、意識の世界で起こる反応には、必ずそれに対応する物質的反応が存在するという考え方である。\cite{wikipedia}
\end{quotation}
クオリアとは、
\begin{quotation}
「感じ」のことである。「イチゴのあの赤い感じ」、「空のあの青々とした感じ」、「二日酔いで頭がズキズキ痛むあの感じ」、「面白い映画を見ている時のワクワクするあの感じ」といった、主観的に体験される様々な質のことである。\par
外部からの刺激(情報)を体の感覚器が捕え、それが神経細胞の活動電位として脳に伝達される。すると何らかの質感が経験される。例えば波長700ナノメートルの光(視覚刺激)を目を通じて脳が受け取ったとき、あなたは「赤さ」を感じる。このあなたが感じる「赤さ」がクオリアの一種である。\cite{wikipediaa}
\end{quotation}
本稿においては、「実験装置の針が1を指していた」、「実験装置の針が2を指していた」という「感じ」もクオリアの一種であると考える。