\subsubsection{意識・クオリアのノーテーション}
本章の最後に、意識・クオリアのノーテションを導入しておこう。まず、意識・クオリアを$|\,\rfloor$の記号で表すことにする。そして、状態$|A\rangle$に意識・クオリアaが随伴していることを、$\mapsto$の記号を用いて、$|A\rangle \mapsto |a \rfloor$と表す。観測問題の焦点は、$|A\rangle \mapsto |a \rfloor$かつ$|B\rangle \mapsto |b \rfloor$のときに、$\alpha|A\rangle + \beta|B\rangle$にどのような意識・クオリアが随伴するのか、ということであるが、マクロな状態のとこの段階ではそれについて示すことは難しいので、後章に回すことにする。
