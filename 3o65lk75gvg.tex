\section{マクロな状態}
本性では、マクロな状態の定義を行う。
\subsection{静的な分割}
宇宙を原子の大きさより十分に大きく、しかし、ほぼ熱平衡に達していると考えられる程度には十分に小さいボリューム$V_p$に分割し、宇宙の状態のヒルベルト空間$\mathcal{H}$を$V_p$の状態のヒルベルト空間$\mathcal{H}_p$ のテンソル積に分割する。すなわち、
\begin{equation}
    \mathcal{H} = \bigotimes_p \mathcal{H}_p
\end{equation}
とする。\\
 次に、各$\mathcal{H}_p$をマクロとミクロのテンソル積へ分割することを考える。すなわち、マクロ変数で指定される状態のヒルベルト空間$\mathcal{H}_p^a$と残りの自由度のうち熱平衡に達していると考えられる自由度のヒルベルト空間$\mathcal{H}_p^t$とで指定される状態のヒルベルト空間$\mathcal{H}_p^i$のテンソル積に分割する。
\begin{equation}
    \mathcal{H}_p = \mathcal{H}_p^a \otimes \mathcal{H}_p^i
\end{equation}
となる。\\
 ハミルトニアン$H$も、以下のように分割する。

E=∑Eja + ∑ Eji + ∑ Ejka + ∑ Ejki + ∑ Ejai

それぞれの意味は、記号から想像されるとおりとする。異なるボリュームでのミクロとマクロの相互作用は、小さいとして無視する。ちなみに、相対論的量子力学を考えているので、近接相互作用しかなく、異なるボリューム間の相互作用は、表面での相互作用である。また、相互作用という表現を使っているが、そのほとんどは流入、流出であって、いわゆる相互作用ではない。