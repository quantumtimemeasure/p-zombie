\section{マクロな状態}
\subsection{静的な分割}
宇宙を原子の大きさより十分に大きく、しかし、熱平衡に達していると考えられる程度には十分に小さいボリューム$V_p$に分割し、宇宙の状態のヒルベルト空間$\mathcal{H}$をVjの状態のヒルベルト空間Hj のテンソル積に分割する。

H=⊗Hj

次に、Hjをマクロとミクロのテンソル積へ分割することを考える。すなわち、マクロ変数で指定される状態のヒルベルト空間Hjaと残りの自由度で指定される状態のヒルベルト空間Hji のテンソル積に分割する。

Hj=Hja ⊗ Hji

ヒルベルト空間と見分けがつかないので、ハミルトニアンをEと書くことにし、Eを以下のように分割する。

E=∑Eja + ∑ Eji + ∑ Ejka + ∑ Ejki + ∑ Ejai

それぞれの意味は、記号から想像されるとおりとする。異なるボリュームでのミクロとマクロの相互作用は、小さいとして無視する。ちなみに、相対論的量子力学を考えているので、近接相互作用しかなく、異なるボリューム間の相互作用は、表面での相互作用である。また、相互作用という表現を使っているが、そのほとんどは流入、流出であって、いわゆる相互作用ではない。