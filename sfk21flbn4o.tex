\subsubsection{温度について}
新解釈では、マクロ・古典的物理変数の固有状態は、ハミルトニアンから、熱平衡にないミクロな状態とマクロ・古典的な状態の相互作用$\sum_pH_p^i$を除いたものによる時間発展では、マクロ・古典的物理変数の固有状態であり続けると考える(これと、随伴現象説の組み合わせが、コペンハーゲン解釈の射影仮説に代わるものである)。マクロ・古典的物理変数には、温度も含まれる。従って、温度についても重ね合わせ状態にはならない。その説明のため、ここでは、宇宙の分割のうち、接触している${V_1,V_2}$のみを考慮することにしよう。また、逆温度以外のマクロ・古典変数は記載を省略することにする。時刻$0$nおいて、$V_1$の逆温度は$\beta_1$
