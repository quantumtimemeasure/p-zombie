\subsubsection{温度について}
新解釈では、マクロ・古典的物理変数の固有状態は、ハミルトニアンから、熱平衡にないミクロな状態とマクロ・古典的な状態の相互作用$\sum_pH_p^i$を除いたものによる時間発展では、マクロ・古典的物理変数の固有状態であり続けると考える(これと、随伴現象説の組み合わせが、コペンハーゲン解釈の射影仮説に代わるものである)。マクロ・古典的物理変数には、温度も含まれる。従って、温度についても重ね合わせ状態にはならない。その説明のため、ここでは、宇宙の分割のうち、接触している${V_1,V_2}$のみを考慮することにしよう。 したがって、
\begin{equation}
    \mathcal{H} = \mathcal{H}_1^a \otimes \mathcal{H}_2^a \otimes \mathcal{H}_1^t \otimes \mathcal{H}_2^t 
\end{equation}
となる。また、逆温度以外のマクロ・古典変数は記載を省略することにする。時刻$0$おいて、$V_1$の逆温度は$\beta_1$、$V_2$の逆温度は$\beta_2$であったとする。時間の進展とともに、両ボリュームの温度は同じ値$\beta$になる。すなわち、
\begin{equation}
    \beta_1(0) = \beta_1,
\end{equation}
\begin{equation}
    \lim_{t \to \infty} \beta_1(t) = \beta,
\end{equation}
\begin{equation}
    \beta_2(0) = \beta_2,
\end{equation}
\begin{equation}
    \lim_{t \to \infty} \beta_2(t) = \beta,
\end{equation}
の関数$\beta_1(t)$、$\beta_2(t)$が存在し、
\begin{equation}
\begin{aligned}
    e^{-iHt}|\beta_1 \rangle_1 \otimes |\beta_2 \rangle_2 \otimes \sum_{v \in \{v_i\}} \sigma_1(v) |v\rangle_1 \otimes \sum_{v \in \{v_i\}}  \sigma_2(v) |v\rangle_2\\
    =  |\beta_1(t) \rangle_1 \otimes |\beta_2(t)\rangle_2 
\end{aligned}
\end{equation}
となる。ただし、簡単のために${H}_1^t = {H}_2^t$とし、その正規直行基底を離散的な\\
 