\subsection{異なる宇宙のボリュームへの分割の重ね合わせについて}
$\{V_1(t)_p\}$と$\{V_2(t)_p\}$という異なる宇宙のボリュームへの分割を用いて表現した状態の重ね合わせ状態を考えることは奇異に感じられるが(異なる分割ではなく、単一の分割でも奇異に感じられるかもしれないが)、これは、場の量子論を正準量子化やフォック空間として捉えるのではなく、場の波動関数として捉えれば、それほど奇異に感じられなくなると思われる。\par
ここでは、簡単に実スカラー場の場合のみを考えよう。そうすると、まず場$\phi$は、空間の点$x \in \mathbb{R}^3$から実数$\mathbb{R}$への写像$\mathbb{R}^3 \to \mathbb{R}$、すなわち、
\begin{equation}
    \phi : \mathbb{R}^3 \ni x \mapsto \phi (x) \in \mathbb{R}
\end{equation}
である。とりあえず二乗可積分の場合のみを考えることにして、$\phi \in L^2(\mathbb{R}^3)$としよう。次に、状態$\ket{\Psi}$は、写像$\phi$から複素振幅$\mathbb{C}$への写像(汎関数)、すなわち、
\begin{equation}
    \label{eq:taotal_universe}
    \ket{\Psi} :  L^2(\mathbb{R}^3) \ni \phi \mapsto \Psi (\phi) \in \mathbb{C}
\end{equation}
である。\par
ボリューム$V_p$の状態とは、ボリューム$V_p$に含まれる点の集合を同じ記号$V_p=\{x \in \mathbb{R}^3 | x \mathrm{\, is\, in\, the\, }V_p\}$で表すことにし、$V_p$上の実数値関数で積分可能なものの全体を$F(V_p)$で表すことにしたとき、写像$\Psi_p : F(V_p) \to \mathbb{C}$である。繰り返しになるが、
\begin{equation}
    \ket{\Psi_p}_{V_p} :  F(V_p) \ni \varphi_p \mapsto \Psi_p (\varphi_p) \in \mathbb{C}
\end{equation}
ということである。\par
宇宙を埋め尽くす空間において、$\Psi_p$が定義されていれば、そこから、\eqref{eq:taotal_universe}の$\Psi$を求めることができる。まず、$\varphi_p$を
\begin{equation}
     \varphi_p(x) = \phi(x),\; x \in V_p \subset \mathbb{R}^3
\end{equation}
と定義し、
\begin{equation}
    \label{eq:universe_wavefunction}
    \Psi (\phi) = \prod_p \Psi_p(\varphi_p)
\end{equation}
とすれば$\Psi (\phi)$が求まるからである。\par
このように、宇宙の分割の方法によらず、宇宙の状態は、$(L^2) = \{\Psi : L^2 \to \mathbb{C}\}$の元として表すことができる。$(L^2)$の元の和は、$(L^2)$に含まれるので、$\{V_1(t)_p\}$と$\{V_2(t)_p\}$という異なる宇宙のボリュームへの分割を用いて表現した状態の重ね合わせ状態も$(L^2)$に含まれる。\par
また、このことから、Euler表示とLagrange表示は相互に変換可能なことがわかるだろう。例えば、Lagrange表示の$\{\Psi_p\}$から\eqref{eq:universe_wavefunction}により$\Psi$を求め、今度はEuler表示の宇宙の分解で、\eqref{eq:universe_wavefunction}を満たす$\{\Psi'_p\}$を見つければ、それがEuler表示の状態になっていると考えられるからである。ただし、このEuler表示は、表示としては一意には定まらないことにも留意が必要である。例えば、$\Psi'_p$を$\alpha$倍して、$\Psi'_q$を$\alpha^{-1}$倍しても$\Psi$は変わらない。しかしこれは、量子力学において一般的にエンタングル状態の表示方法は一意に定まらないのと同じことであり、同じ状態を様々に表示できるということである。簡単に電子のスピンの状態で記載すると、
\begin{equation}
    \ket{\uparrow}=\frac{1}{\sqrt{2}}\ket{\leftarrow}+\frac{i}{\sqrt{2}}\ket{\rightarrow},
\end{equation}
\begin{equation}
    \ket{\downarrow}=\frac{1}{\sqrt{2}}\ket{\leftarrow}-\frac{i}{\sqrt{2}}\ket{\rightarrow}
\end{equation}
とすると、
\begin{equation}
    \frac{1}{\sqrt{2}}\ket{\uparrow} \otimes \ket{\downarrow} + \frac{1}{\sqrt{2}}\ket{\downarrow} \otimes \ket{\uparrow}
    = 
    \frac{1}{\sqrt{2}}\ket{\leftarrow} \otimes \ket{\rightarrow} + \frac{1}{\sqrt{2}}\ket{\rightarrow} \otimes \ket{\leftarrow}
\end{equation}
である。このように、表示方法には任意性がある。新解釈では、古典・マクロな物理量に関しては、この表示の任意性はないと考える。