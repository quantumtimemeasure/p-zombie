\subsubsection{熱力学の第二法則が成り立つ理由}
観測問題という本稿のテーマとは少し離れるが、ここで、熱力学の第二法則、すなわち熱が温度の高い方から低い方に流れる理由について、仮説を述べておこう。熱力学の第二法則は、一般的に時間反転対称性を破っていると考えられているが、新解釈では、量子力学が正しいとしているため、時間反転対称性は成り立っていると考える。時間反転対称性は、
\begin{equation}
    \beta_1(t) = \beta_1(-t),
\end{equation}
\begin{equation}
    \beta_2(t) = \beta_21(-t),
\end{equation}
\begin{equation}
    \sigma(t,v,w) = \sigma(-t,v,w)
\end{equation}
であれば、成り立つとかんがえあれr
