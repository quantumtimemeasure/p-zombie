\subsubsection{熱力学の第二法則が成り立つ理由}
観測問題という本稿のテーマとは少し離れるが、ここで、熱力学の第二法則、すなわち熱が温度の高い方から低い方に流れる理由について、仮説を述べておこう。熱力学の第二法則は、一般的に時間反転対称性を破っていると考えられているが、新解釈では、量子力学が正しい(量子電磁力学QEDは時間反転対称性を持つ)としているため、時間反転対称性は成り立っていると考える。時間反転演算子を$T$として、時間反転対称性は、
\begin{equation}
\label{eq:temperature_synmetry}
    \beta_p(t) = \beta_p(-t), \; p \in \{1,2\},
\end{equation}
\begin{equation}

\label{eq:time_reversal}
    T|\beta_1(t) \rangle_1 \otimes |\beta_2(t)\rangle_2 \otimes   \sum_{v \in \{v_i\} ,\, w \in \{v_i\}} \sigma(t,v,w)|v\rangle_1 \otimes |w\rangle_2 
    =|\beta_1(t) \rangle_1 \otimes |\beta_2(t)\rangle_2 \otimes   \sum_{v \in \{v_i\} ,\, w \in \{v_i\}} \sigma(-t,v,w)|v\rangle_1 \otimes |w\rangle_2
\end{equation}
であれば、成り立つと考えられる。温度は時間反転により変わらないと考えられるため(例えば、気体の分子の速度を全て反転しても温度は変わらないと考えられているため)、全ての$t$について、$T|\beta_p(t) \rangle_p = |\beta_p(t) \rangle_p$としている。ここで、($\ref{eq:temperature_synmetry}$)のように$\beta_p(t) = \beta_p(-t)$であれば、時刻$t$の状態を時間反転した状態、すなわち($\ref{eq:time_reversal}$)は、$|\beta_1(-t) \rangle_1 \otimes |\beta_2(-t)\rangle_2 \otimes \sum_{v \in \{v_i\} ,\, w \in \{v_i\}} \sigma(-t,v,w)|v\rangle_1 \otimes |w\rangle_2$となる。これは、時刻$-t$での状態である。従って、当該時刻から時間が$t$進むと、時刻$0$の状態となる。従って、時間反転した状態であることが確かめられる。より一般的には、時刻$t$の状態を$|t\rangle$、それを$\ref{eq:time_reversal}$により時間反転した状態を$|Tt\rangle$と書けば、
\begin{equation}
    e^{-iHt'}|Tt\rangle = |t-t'\rangle
\end{equation}
となっていることが確かめられる。これは、正しく時間反転状態となっているということである。
  逆は必ずしも真ではないので、時間反転状態が$ref{eq:time_reversal}$であるとは限らないが、時間反転についての考察は本稿の主題ではないので、本稿では$ref{eq:time_reversal}$が時間反転状態であると考えることにする。
  以上の考察から、時刻$0$において接触した状態に、時刻$t<0$から接触したままの時間発展でなったとすると、それまでの間、すなわち$t<0$において、接触した領域間の温度差は広がっていくことになる。$\beta_p(t) = \beta_p(-t)$だからである。これが理論的な考察結果であるが、残念ながら、それを実験で確かめることはできない。時刻$0$に接触させる以外に、時刻$0$と同じ状態を準備する方法を我々は知らないからである。ハミルトニアンによる時間発展においては、時間反転の対称性は保たれているが、それを確かめる実験を行うことはできないというのが、熱力学の第二法則が時間反転対称性を破っているようにみえる理由である。
  時刻$0$の特殊性は、$\mathcal{H}_1^a \otimes \mathcal{H}_2^a \otimes \mathcal{H}_1^t \otimes \mathcal{H}_2^t$というテンソル積で表されるヒルベルト空間において、エンタングルのまったくない状態ということである。この時に、温度差は最大になる。一方、$0$以外の時刻では、$\sum_{v \in \{v_i\} ,\, w \in \{v_i\}} \sigma(t,v,w)|v\rangle_1 \otimes |w\rangle_2 $の式が示すようにエンタングル状態にある。このことから、エンタングルのない状態から時間発展又は時間後進すると温度差が小さくなると想像される。
  別の観点から熱力学の第二法則が成り立つ理由を理解しようとするならば、ヒルベルト空間$\mathcal{H}$には、時間発展に伴い温度差が縮小する部分空間$\mathcal{H}^+$、時間発展に伴い温度差が拡大する部分空間$\mathcal{H}^-$、もともと温度差がなく時間発展で温度差が生じない部分空間$\mathcal{H}^=$があり(多分、$\mathcal{H}=\mathcal{H}^+ \cup \mathcal{H}^- \cup \mathcal{H}^=$であり)、熱力学の第二法則は、「人間が始状態として用意できるのは部分空間$\mathcal{H}^+$の要素だけである。」と表すことができるだろう。ただし、ここで完全に同じ温度の状態(誤差のない状態)を用意することはできないので$\mathcal{H}^=$を用意することはできないとした。時間発展により有限時間で厳密に$\mathcal{H}^=$の要素に状態がなることもないと考えられる。
  そうすると、次の疑問は、なぜ人間には$\mathcal{H}^+$の状態しか用意できないのかということになる。これは、観測した状態しか操作できないためと思われる。観測していない物体を人は動かすことができないことは、ある意味当然であろう(哲学としては、その理由が不思議で検討する必要があるということになり、哲学である本稿はそれを検討する必要があるという立場であるが、ここではそれについては不問にしておきたい)。どこにあるかも把握していない(=観測していない)物体を動かすことはできない。したがって、「なぜ人間には$\mathcal{H}^+$の状態しか用意できないのか?」という疑問は、「量子力学における観測とは何か?」「観測によって波束が収束するのはなぜか?」という量子力学の観測問題と極めて強く関わってくる。したがって、本稿の主題である観測問題、新解釈の全体像を示さなければ、「なぜ人間には$\mathcal{H}^+$の状態しか用意できないのか?」という疑問に答えることはできない。そのため、この疑問には、新解釈の全体像を示したのちに改めて答えを検討することにしたい。