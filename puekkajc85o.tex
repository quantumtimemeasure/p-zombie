\subsubsection{意識・クオリアの相互作用表示の具体的イメージ例}
意識・クオリアにかかる相互作用表示をイメージすることは難しいため、ここでは、その例を具体的に示してみることにしたい。意識・クオリアがどのように物理的な状態に随伴しているかは、脳科学で現在のところ十分に解明されていないため、ここでの記載は、あくまで可能性の一つである。\\
 例えば、\cite{Murray_2016}では、意識・クオリアの実体が、脳細胞の発火率の主成分であることが示唆されている。これは、神経細胞$i$の平均をゼロに調整した発火率を$r_i$とすると、ある$w_k^i$があって、意識・クオリアは$w_k^ir_i$に随伴しているということと理解できるだろう。ただし、ここで、$k\in\{1,2,\ldots,K\}$であり、同じ添字$i$については和を取るものとする。\\
 すなわち、主成分の値の集合$P=\mathbb{R}^K$の分割$C_j,\,j\in\{0,1,2,\ldots,N\},\, P=\bigcup_j C_j,\, C_j \cap C_{j'}=\emptyset \, (j \neq j')$があって、$w_k^ir_i \in C_j$において、$j=0$の場合は意識・クオリアは随伴しておらず、$j \neq 0$の場合は、意識・クオリア$j$が随伴しているということである。ただしここでは、人間は有限の$N$種類の意識・クオリアを持ち得ると仮定した。\\
 相互作用表示を用いたい趣旨は、各$C_j$の代表点$c_j \in C_j$をとって、物理的な状態を$|c_j,other \ physical \ quantities \rangle$のように書きたいということである。$other \ physical \  quantities$には、$w_k^ir_i - c_{j,k}$の値が含まれることになる。ここで、$c_{j,k}$は、$c_j \in \mathbb{R}^K$の第$k$成分の値である。\\
 意識・クオリアの随伴にとって内在的な$H_c$では、随伴している意識・クオリアが変わることはないため、シュレディンガー表示においても
\begin{equation}
  e^{-iH_ct}|c_j,others \rangle = |c_j,others' \rangle 
\end{equation}
である。時間発展$e^{-iH_ct}$によって、意識・クオリアが異なる状態にはならないからである(意識・クオリアがワーキングメモリに基づくものだとすると、時間的に数秒しか継続しないため、時間の経過とともに $|c_0,others' \rangle \notin \mathfrak{C}$の状態になる必要があり、その変化は$e^{-iH_ot}$によりもたらされると考えることにする)。\\
 しかし、各神経細胞$i$の位置$p$の電位の組$\{E_{i,p}\}$によって随伴する意識・クオリアは決まっていると考え、その状態を$|\{ E_{i,p}\},others \rangle$と記載することにするなら、シュレディンガー表示では同じ意識・クオリアが継続している間も、時間により状態は異なる$|\{ E'_{i,p}\},others' \rangle$になってしまう。一方、相互作用表示では、同じ意識・クオリアが継続している間は、電位の組$\{E_{i,p}\}$については同じ状態$|\{ E_{i,p}\},others' \rangle$と記載することができる。これが、相互作用表示を用いる理由である。\\
 発火率の