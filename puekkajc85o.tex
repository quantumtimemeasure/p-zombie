\subsubsection{意識・クオリアの相互作用表示の具体的イメージ例}
意識・クオリアにかかる相互作用表示をイメージすることは難しいため、ここでは、その例を具体的に示してみることにしたい。意識・クオリアがどのように物理的な状態に随伴しているかは、脳科学で現在のところ十分に解明されていないため、ここでの記載は、あくまで可能性の一つである。\\
 例えば、\cite{Murray_2016}では、意識・クオリアの実体が、脳細胞の発火率の主成分であることが示唆されている。これは、神経細胞$i$の平均をゼロに調整した発火率を$r_i$とすると、ある$w_k^i$があって、意識・クオリアは$w_k^ir_i$に随伴しているということと理解できるだろう。ただし、ここで、$k\in\{1,2,\ldots,K\}$であり、同じ添字$i$については和を取るものとする。\\
 すなわち、主成分の値の集合$P=\mathbb{R}^K$の分割$C_j,\,j\ P=\bigcup_j C_j,\, C_j \cap C_{j'}=\emptyset \, (j \neq j')$があって、$w_k^ir_i \in C_j$においては、意識・クオリアjが随伴しているということである。
 